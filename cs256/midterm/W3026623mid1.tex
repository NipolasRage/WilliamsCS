\documentclass[10pt]{article}

% DO NOT EDIT THE LINES BETWEEN THE TWO LONG HORIZONTAL LINES

%---------------------------------------------------------------------------------------------------------

% Packages add extra functionality.
\usepackage{times,graphicx,epstopdf,fancyhdr,amsfonts,amsthm,amsmath,algorithm,algorithmic,xspace,hyperref}
\usepackage[left=1in,top=1in,right=1in,bottom=1in]{geometry}
\usepackage{sect sty}	%For centering section headings
\usepackage{enumerate}	%Allows more labeling options for enumerate environments 
\usepackage{epsfig}
\usepackage[space]{grffile}

% This will set LaTeX to look for figures in the same directory as the .tex file
\graphicspath{.} %The dot means current directory.

\pagestyle{fancy}

\lhead{\WilliamsID}
\chead{Midterm \ --- Problem \ProblemNumber}
\rhead{\today}
\lfoot{CSci 256: Algorithm Design}
\cfoot{\thepage}
\rfoot{Spring 2018}

% Some commands for changing header and footer format
\renewcommand{\headrulewidth}{0.4pt}
\renewcommand{\headwidth}{\textwidth}
\renewcommand{\footrulewidth}{0.4pt}

% These let you use common environments
\newtheorem{claim}{Claim}
\newtheorem{definition}{Definition}
\newtheorem{theorem}{Theorem}
\newtheorem{lemma}{Lemma}
\newtheorem{observation}{Observation}
\newtheorem{question}{Question}

\setlength{\parindent}{0cm}


%---------------------------------------------------------------------------------------------------------

% DON'T CHANGE ANYTHING ABOVE HERE

% Edit below as instructed
\newcommand{\WilliamsID}{W3026623}	% Put you Williams ID in the braces
\newcommand{\ProblemNumber}{1}		% Put the problem # in the braces
\newcommand{\ProblemHeader}{Problem \ProblemNumber}	% Don't change this

\begin{document}


\textbf{Solution}
\\a) Claim: We want to show that if a $DFS$ tree $T_{1}$ and a $BFS$ tree $T_{2}$ of a connected, undirected graph $G = (V,E)$ are equal, then $G = T_{1} = T_{2} = T$.
\begin{proof}
	Suppose, for the sake of contradiction, the claim is false. That is, $G \ne T$ even if $T_{1} = T_{2}$. This means that there exists an edge $e$ that is in $G$ but does not belong to $T$.	A $BFS$ traverses the vertices of a graph a level at a time. Meaning that vertices at level $i$ are of distance $i$ from the root. Thus, an edge $e$ in $G$ that is not in $T$ connects vertices at consecutive levels or at the same level of $T_{2}$. On the other hand, a $DFS$ traversal of $G$ would follow the path of $e$ up or accross the level considered by the $BFS$ traversal. This is because $DFS$ traverses down edges until it reaches an end before going back on any other unexplored path. Therefore, $T_{1} \ne T_{2}$. We have reached a contradiction because it is given that $T_{1} = T_{2}$. We have shown that if a $DFS$ tree is the same as a $BFS$ tree then the graph is equal to said tree.
	
\end{proof}

%End of feedback section

% DO NOT DELETE ANYTHING BELOW THIS LINE
\end{document}
