\documentclass[10pt]{article}

% DO NOT EDIT THE LINES BETWEEN THE TWO LONG HORIZONTAL LINES

%---------------------------------------------------------------------------------------------------------

% Packages add extra functionality.
\usepackage{times,graphicx,epstopdf,fancyhdr,amsfonts,amsthm,amsmath,algorithm,algorithmic,xspace,hyperref}
\usepackage[left=1in,top=1in,right=1in,bottom=1in]{geometry}
\usepackage{sect sty}	%For centering section headings
\usepackage{enumerate}	%Allows more labeling options for enumerate environments 
\usepackage{epsfig}
\usepackage[space]{grffile}
\usepackage{listings}
% This will set LaTeX to look for figures in the same directory as the .tex file
\graphicspath{.} %The dot means current directory.

\pagestyle{fancy}

\lhead{\WilliamsID}
\chead{Midterm \ --- Problem \ProblemNumber}
\rhead{\today}
\lfoot{CSci 256: Algorithm Design}
\cfoot{\thepage}
\rfoot{Spring 2018}

% Some commands for changing header and footer format
\renewcommand{\headrulewidth}{0.4pt}
\renewcommand{\headwidth}{\textwidth}
\renewcommand{\footrulewidth}{0.4pt}

% These let you use common environments
\newtheorem{claim}{Claim}
\newtheorem{definition}{Definition}
\newtheorem{theorem}{Theorem}
\newtheorem{lemma}{Lemma}
\newtheorem{observation}{Observation}
\newtheorem{question}{Question}

\setlength{\parindent}{0cm}


%---------------------------------------------------------------------------------------------------------

% DON'T CHANGE ANYTHING ABOVE HERE

% Edit below as instructed
\newcommand{\WilliamsID}{W3026623}	% Put you Williams ID in the braces
\newcommand{\ProblemNumber}{4}		% Put the problem # in the braces
\newcommand{\ProblemHeader}{Problem \ProblemNumber}	% Don't change this

\begin{document}


\textbf{Solution}
\\a) Let $opt(j)$ be the number of triangulations of a convex $j-$gon. We know that polygon edge $p_{j}p_{1}$ must form a triangle with $p_{i}$ for some $i$, where $1<i<j$. Our base case would be if $j<4$ then $opt(j) = 1$. We then take $i$ from $2\rightarrow j-1$ and we have: $opt(j) = \sum_{i=2}^{j-1}\mbox{ if } i < 4 \mbox{ then } opt(j-i+1) \mbox{ else if } j-i<3 \mbox{ then } opt(i) \mbox{ else } opt(j-i+1) + opt(i)$.
\begin{lstlisting}
def opt(j):
	r=0
	if (j<4):
		return 1
	else:
		for i in range(2,j):
			if i < 4:
				r+= opt(j-i+1)
			elif j-i<3:
				r+= opt(i)
			else:
				r += opt(j-i+1)+opt(i)
		return r
\end{lstlisting}
Time Complexity: All possible $i: O(j) *$ At most $j$ recursive calls per iteration: $O(j)=O(j^{2})$
\begin{proof}
	We will prove that this dynamic programming algorithm works for all $j-gon$.\\
	Base Case: Let $j=3$. Triangles only have one element in their triangulation set as they have $3-3=0$ crossing diagonals. Our algorithm computes 1 for $j=3$, so it holds for this base case.\\
	Inductive Hypothesis: Assume our algorithm computes the right number of triangulations up to a $k-gon$, where $k\ge 3$.\\
	Inductive Step: We must show that our algorithm computes the correct number of triangulations for a $k+1-gon$. It is given that polygon edge $p_{k+1}p_{1}$ must form a triangle with $p_{i}$ for some $i$, where $1<i<k+1$. Our algorithm goes through every possible value of $i$. When $i=2$, we have a diagonal from $p_{k+1}$ to $p_{2}$. This leaves us with a triangle $p_{k+1}p_{1}p_{2}$ and a $k-gon$. From our inductive hypothesis, we know our algorithm provides the correct number of possible triangulations for a $k-gon$, we only add this number as triangles have one possible triangulation. Similarly, when $i=3$ we add the triangulations of the $k-1-gon$ found by our algorithm based on our IH. A similar thing happens when $i$ approaches $k+1$. Triangles are made between $p_{1}p_{k+1}p_{i}$ and the $k+1-gon$ is divided into at least one triangle and a $k-gon$ or $k-1-gon$. We use our IH to add the possible triangulations of these two shapes. Finally, the triangle $p_{1}p_{k+1}p_{i}$ divides the $k+1-gon$ into two shapes of less than $k+1$ sides. From our IH we take the sum of triangulations of both polygons and ignore the triangle. By splitting our $k+1-gon$ into smaller shapes, we are able to use our IH to calculate the number of triangulations of a $k+1-gon$.
\end{proof}
b) Let $opt(j,a,z)$ be the weight of a minimum weight triangulation of $P$. It is obvious that the weight of the minimum triangulation of a triangle is zero, because the triangulation contains no edges. We will use $w_{j,i}$ to denote the weight of the edge between $p_{j}p_{i}$. If said edge is not a part of a triangulation $T$, then $w_{j,i}=0$. We will walk through vertices $p_{2}\rightarrow p_{j-1}$, adding the weight of the edges of the triangle formed by $p_{j}p_{i}p_{1}$, and any other edge found in the shapes found after this split. We take the minimum weight we find after all possible triangulations. We can build the following recurrence: \\$opt(j,a,z) = \min_{a<i<z}\{w_{a,i}+w_{z,i}+opt(j-i+1,i,z)+opt(i,a,i)\}$ where $a$ and $z$ will be initialized to $1$ and $j$ respectively. \\
Time Complexity: All possible $i:O(j)*j$ recursive calls per iteration = $O(j^2)$.\\
Space Complexity: Array of size $j: O(j)$.
\begin{proof}
	We will proof that our dynamic programming algorithm finds the weight of a minimum weight triangulation of $P$.\\
	Base Case: A triangle has a minimum weight triangulation of zero because its triangulation has no edges. Our algorithm will return zero, so it works for this case.\\
	Inductive Hypothesis: Assume our algorithm returns the weight of a minimum weight triangulation of a convex polygon with up to $k$ sides, where $k\ge3$.\\
	Inductive Step: We must show our algorithm returns the desired output for a convex polygon $P^{*}$ with $k+1$ sides. As we use $i$ from 2 to $k$, we add the weight of the edge from $p_{1}$ to $p_{i}$ and from $p_{k+1}$ to $p_{i}$ to our total weight for this triangulation (Note that edges not in a triangulation of $P^{*}$ have a weight of zero). These edges divide $P^{*}$ into at most three polygons of less than $k+1$ edges. Using our Inductive Hypothesis we are able to determine the weight of a minimum weight triangulation of these shapes. Once these minimum weights are added to the weight of the edges connected to $i$, we increase $i$ until it reaches $k-1$. We were now able to compare all of the weights of all triangulations of $P^{*}$ and can easily determine the smallest one. Thus, our algorithm can find a correct weight of a minimum weight triangulation of any convex polygon. 
\end{proof}
c) 
An $opt[−]$ array can be filled as the algorithm is running so $O(j^{2})$ time and $O(j)$ space. Therefore, this modification only adds space complexity. From the array, one can extract a minimum weight triangulation $M$ of $P$ by moving backward through the array: Starting with $i=2$, $edge_{j,2}\in M$  if $opt[j]=w_{1,2}+w_{j,2}+opt(j-2+1,2,j)+opt(2,1,2)$. Letting $i=3,4,...,j-1$ then $edge_{j,i} \mbox{ and }edge_{1,i}\in M$ assuming  $opt[j]=w_{1,i}+w_{j,i}+opt(j-i+1,i,j)+opt(i,1,i)$ and $edge_{j,i-1},edge_{j,i+1},edge_{1,i-1},edge_{1,i+1}\not\in M$; otherwise $edge_{j,i},edge_{1,i}\not\in M$. Looking this up will add $O(j)$ to run time but will not affect the asymptotic run time of the algorithm.  
%End of feedback section

% DO NOT DELETE ANYTHING BELOW THIS LINE
\end{document}
