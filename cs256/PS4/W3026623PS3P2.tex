\documentclass[10pt]{article}

% DO NOT EDIT THE LINES BETWEEN THE TWO LONG HORIZONTAL LINES

%---------------------------------------------------------------------------------------------------------

% Packages add extra functionality.
\usepackage{times,graphicx,epstopdf,fancyhdr,amsfonts,amsthm,amsmath,algorithm,algorithmic,xspace,hyperref}
\usepackage[left=1in,top=1in,right=1in,bottom=1in]{geometry}
\usepackage{sect sty}	%For centering section headings
\usepackage{enumerate}	%Allows more labeling options for enumerate environments 
\usepackage{epsfig}
\usepackage[space]{grffile}

% This will set LaTeX to look for figures in the same directory as the .tex file
\graphicspath{.} %The dot means current directory.

\pagestyle{fancy}

\lhead{\WilliamsID}
\chead{Problem Set \PSNumber \ --- Problem \ProblemNumber}
\rhead{\today}
\lfoot{CSci 256: Algorithm Design}
\cfoot{\thepage}
\rfoot{Spring 2018}

% Some commands for changing header and footer format
\renewcommand{\headrulewidth}{0.4pt}
\renewcommand{\headwidth}{\textwidth}
\renewcommand{\footrulewidth}{0.4pt}

% These let you use common environments
\newtheorem{claim}{Claim}
\newtheorem{definition}{Definition}
\newtheorem{theorem}{Theorem}
\newtheorem{lemma}{Lemma}
\newtheorem{observation}{Observation}
\newtheorem{question}{Question}

\setlength{\parindent}{0cm}


%---------------------------------------------------------------------------------------------------------

% DON'T CHANGE ANYTHING ABOVE HERE

% Edit below as instructed
\newcommand{\WilliamsID}{W3026623}	% Put you Williams ID in the braces
\newcommand{\PSNumber}{4}			% Put the problem set # in the braces
\newcommand{\ProblemNumber}{2}		% Put the problem # in the braces
\newcommand{\ProblemHeader}{Problem \ProblemNumber}	% Don't change this

\begin{document}

\vspace{\baselineskip}	% Add some vertical space
\textbf{I collaborated with:} 


\vspace{\baselineskip}	% Add some vertical space
\textbf{Problem}

Suppose you are choosing between the following three algorithms:
\begin{enumerate}
	\item Algorithm A solves problems by dividing them into five subproblems of half the size, recursively solving each subproblem, and then combining the solutions in linear time.
	\item Algorithm B solves problems of size $n$ by recursively solving two subproblems of size $n-1$ and then combining the solutions in constant time.
	\item Algorithm C solves problems of size $n$ by dividing them into nine subproblems of size $n/3$, recursively solving each subproblem, and then combining the solutions in $O(n^{2})$ time.
\end{enumerate}
What are the running times of each of these algorithms (in asymptotic notation) and which would you choose?\\
\textbf{Solution}
\begin{enumerate}
	\item $T(n)=5T(\frac{n}{2})+cn = O(n^{\log _{2}5})$
	\item $T(n)=2T(n-1)+c = O(2^{n})$ as every step makes two recursive calls of size $n-1$. If we think about this as a tree, $n-1$ eventually approaches zero and $2^{n}$ leaves are created. 
	\item $T(n)=9T(\frac{n}{3}) + cn^{2} = O(n^{2}\log n)$
\end{enumerate}
I would choose Algorithm C as $2^{n}>n^{\log _{2}5}$ because exponentials grow quicker than polynomials and $n^{\log _{2}5} > n^{2}\log n$ as:\\
\begin{gather*}
\lim_{n\to\infty}\frac{n^{2}\log n}{n^{\log _{2}5}}\\
\lim_{n\to\infty}\frac{\log n}{n^{0.32}} = 0
\end{gather*}
Polynomials grow quicker than logarithmic functions.
%End of feedback section

% DO NOT DELETE ANYTHING BELOW THIS LINE
\end{document}
