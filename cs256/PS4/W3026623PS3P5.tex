\documentclass[10pt]{article}

% DO NOT EDIT THE LINES BETWEEN THE TWO LONG HORIZONTAL LINES

%---------------------------------------------------------------------------------------------------------

% Packages add extra functionality.
\usepackage{times,graphicx,epstopdf,fancyhdr,amsfonts,amsthm,amsmath,algorithm,xspace,hyperref}
\usepackage[left=1in,top=1in,right=1in,bottom=1in]{geometry}
\usepackage{sect sty}	%For centering section headings
\usepackage{enumerate}	%Allows more labeling options for enumerate environments 
\usepackage{epsfig}
\usepackage[space]{grffile}
\usepackage{algpseudocode}

% This will set LaTeX to look for figures in the same directory as the .tex file
\graphicspath{.} %The dot means current directory.

\pagestyle{fancy}

\lhead{\WilliamsID}
\chead{Problem Set \PSNumber \ --- Problem \ProblemNumber}
\rhead{\today}
\lfoot{CSci 256: Algorithm Design}
\cfoot{\thepage}
\rfoot{Spring 2018}

% Some commands for changing header and footer format
\renewcommand{\headrulewidth}{0.4pt}
\renewcommand{\headwidth}{\textwidth}
\renewcommand{\footrulewidth}{0.4pt}

% These let you use common environments
\newtheorem{claim}{Claim}
\newtheorem{definition}{Definition}
\newtheorem{theorem}{Theorem}
\newtheorem{lemma}{Lemma}
\newtheorem{observation}{Observation}
\newtheorem{question}{Question}

\setlength{\parindent}{0cm}


%---------------------------------------------------------------------------------------------------------

% DON'T CHANGE ANYTHING ABOVE HERE

% Edit below as instructed
\newcommand{\WilliamsID}{W3026623}	% Put you Williams ID in the braces
\newcommand{\PSNumber}{4}			% Put the problem set # in the braces
\newcommand{\ProblemNumber}{5}		% Put the problem # in the braces
\newcommand{\ProblemHeader}{Problem \ProblemNumber}	% Don't change this

\begin{document}

\vspace{\baselineskip}	% Add some vertical space
\textbf{I collaborated with:} 


\vspace{\baselineskip}	% Add some vertical space
\textbf{Problem}

Solve Problem 6 from Chapter 5 of your text.  \textbf{Clarification}.  What does {\em probe} mean?  Imagine that for each vertex $v$ in the tree, the label $x_v$ is a quantity that must be computed and that is very time-consuming to compute $x_v$.
A {\em probe} of $v$ is a computation of $x_v$ so we want to minimize the number of such operations.\\
\textbf{Solution}\\
\begin{algorithm*}[h] 
	\caption{Find local minimum} 
	\begin{algorithmic}
		\Function{min}{$T$}
		\If {$T$ is a leaf}
		\State \Return $T.value$
		\ElsIf {$T.value$ is less than both of its children}
		\State \Return $T.value$
		\Else
		\State $min(smallest\ child)$
		\EndIf
		\EndFunction
	\end{algorithmic}
\end{algorithm*} 
\begin{proof}
	If $T$ is a leaf then we know no edge is incident of it, so it is a local minimum. Moreover, if $T$ has children, then we check if its smaller than them. We do not have to check its parent as our recursive call handles it. If $T$ happens to be bigger than its children, then we know the children are smaller and do not have to check for the parent again. We keep going down the tree until we reach a leaf or a node that is a local minimum. Note that we only move down the tree if there is a smaller node.\\
	This takes $O(\log n)$ as the most traversing we do is all the way to a leaf and the depth of a tree is $\log n$
\end{proof}
%End of feedback section

% DO NOT DELETE ANYTHING BELOW THIS LINE
\end{document}
