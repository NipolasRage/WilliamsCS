\documentclass[10pt]{article}

% DO NOT EDIT THE LINES BETWEEN THE TWO LONG HORIZONTAL LINES

%---------------------------------------------------------------------------------------------------------

% Packages add extra functionality.
\usepackage{times,graphicx,epstopdf,fancyhdr,amsfonts,amsthm,amsmath,algorithm,xspace,hyperref}
\usepackage[left=1in,top=1in,right=1in,bottom=1in]{geometry}
\usepackage{sect sty}	%For centering section headings
\usepackage{enumerate}	%Allows more labeling options for enumerate environments 
\usepackage{epsfig}
\usepackage[space]{grffile}
\usepackage{algpseudocode}

% This will set LaTeX to look for figures in the same directory as the .tex file
\graphicspath{.} %The dot means current directory.

\pagestyle{fancy}

\lhead{\WilliamsID}
\chead{Problem Set \PSNumber \ --- Problem \ProblemNumber}
\rhead{\today}
\lfoot{CSci 256: Algorithm Design}
\cfoot{\thepage}
\rfoot{Spring 2018}

% Some commands for changing header and footer format
\renewcommand{\headrulewidth}{0.4pt}
\renewcommand{\headwidth}{\textwidth}
\renewcommand{\footrulewidth}{0.4pt}

% These let you use common environments
\newtheorem{claim}{Claim}
\newtheorem{definition}{Definition}
\newtheorem{theorem}{Theorem}
\newtheorem{lemma}{Lemma}
\newtheorem{observation}{Observation}
\newtheorem{question}{Question}

\setlength{\parindent}{0cm}
\usepackage{tikz}
\usepackage{enumitem}
%---------------------------------------------------------------------------------------------------------

% DON'T CHANGE ANYTHING ABOVE HERE

% Edit below as instructed
\newcommand{\WilliamsID}{W3026623}	% Put you Williams ID in the braces
\newcommand{\PSNumber}{5}			% Put the problem set # in the braces
\newcommand{\ProblemNumber}{2}		% Put the problem # in the braces
\newcommand{\ProblemHeader}{Problem \ProblemNumber}	% Don't change this

\begin{document}

\vspace{\baselineskip}	% Add some vertical space
\textbf{I collaborated with:} 


\vspace{\baselineskip}	% Add some vertical space
\textbf{Problem}

Answer Exercise 6 (page 317) in Chapter 6 of your text.  Include the requisite justification of correctness and complexity analysis.  Also address the following: Why do you think the problem has you minimize the sum of the {\em squares} of the slack of each line, as opposed to the the sum of the slacks themselves or of the absolute values of the slacks.

By the way, in the sentence ``The difference between the left-hand side and the right-hand side will be called the {\em slack} of the line..." the authors are referring to the left-hand side and the right-hand side of the inequality.

Hint: Wouldn't it be useful to know all possible potential slack values...?\\
\textbf{Solution}\\
Let $S_{i,j}$ denote the slack of a line starting at word $w_{i}$ and ending at word $w_{j}$. We must start from the last line and figure out how many words appear on this line. We start from word $w_{n}$ and check to see up to what word $w_{k}$ we can fill this line where $k \le n$. Note that $S_{k,n}$ takes $O(n)$ and we must find the minimum slack for said line. Once we find it we move on to find $S_{i,k-1}$ where $i \le k-1$. We can build the following recurrence: $opt[n] = \min _{1\le k \le n} (S_{k,n}^{2} + opt[k-1])$. 
\begin{algorithm*}[h] 
	\caption{Determines partition of words} 
	\begin{algorithmic}
		\State $opt[0] = 0$
		\For {i from 1 to n}
		\State $opt[i] = \min _{1\le k \le i} (S_{k,i}^{2} + opt[k-1]) $
		\EndFor
	\end{algorithmic}
\end{algorithm*}
\\We then simply trace back $opt$ and find the optimal solution for an additional $O(n)$. The problem asks us to minimize the sum of the squares of slack because it emphasizes on more spaces for words on a line.\\Time complexity: $O(n^{2})$ \\Space Complexity: $O(n)$\\
\begin{proof}
	Base Case: If the text consist of a sequence of 1 word, then that word should constitute the last line. Our algorithm will find the slack of this word and add $opt[0] = 0$ to it. This is the minimum possible slack for this sequence, so our algorithm works for this case.\\
	Inductive Hypothesis: Assume our algorithm solves the problem for a sequence up to $k$ words where $k\ge 1$.\\
	Inductive Step: We must now show that our algorithm provides the best solution for a sequence of size $k+1$. Let $W$ be a sequence of words of size $k+1$. We now compute $S_{j,k+1}$ where $j\le k+1$. To this we add the optimal slack of the remaining sets of words. These sets can be found because they are of size less than $k+1$, and our inductive hypothesis assures this number is minimized. We then can figure out the minimum $S_{k,i}^{2} + opt[k-1]$. Thus, our algorithm works.
\end{proof}

%End of feedback section

% DO NOT DELETE ANYTHING BELOW THIS LINE
\end{document}
