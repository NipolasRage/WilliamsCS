\documentclass[10pt]{article}

% DO NOT EDIT THE LINES BETWEEN THE TWO LONG HORIZONTAL LINES

%---------------------------------------------------------------------------------------------------------

% Packages add extra functionality.
\usepackage{times,graphicx,epstopdf,fancyhdr,amsfonts,amsthm,amsmath,algorithm,xspace,hyperref}
\usepackage[left=1in,top=1in,right=1in,bottom=1in]{geometry}
\usepackage{sect sty}	%For centering section headings
\usepackage{enumerate}	%Allows more labeling options for enumerate environments 
\usepackage{epsfig}
\usepackage[space]{grffile}
\usepackage{algpseudocode}

% This will set LaTeX to look for figures in the same directory as the .tex file
\graphicspath{.} %The dot means current directory.

\pagestyle{fancy}

\lhead{\WilliamsID}
\chead{Problem Set \PSNumber \ --- Problem \ProblemNumber}
\rhead{\today}
\lfoot{CSci 256: Algorithm Design}
\cfoot{\thepage}
\rfoot{Spring 2018}

% Some commands for changing header and footer format
\renewcommand{\headrulewidth}{0.4pt}
\renewcommand{\headwidth}{\textwidth}
\renewcommand{\footrulewidth}{0.4pt}

% These let you use common environments
\newtheorem{claim}{Claim}
\newtheorem{definition}{Definition}
\newtheorem{theorem}{Theorem}
\newtheorem{lemma}{Lemma}
\newtheorem{observation}{Observation}
\newtheorem{question}{Question}

\setlength{\parindent}{0cm}
\usepackage{tikz}
\usepackage{enumitem}
%---------------------------------------------------------------------------------------------------------

% DON'T CHANGE ANYTHING ABOVE HERE

% Edit below as instructed
\newcommand{\WilliamsID}{W3026623}	% Put you Williams ID in the braces
\newcommand{\PSNumber}{7}			% Put the problem set # in the braces
\newcommand{\ProblemNumber}{2}		% Put the problem # in the braces
\newcommand{\ProblemHeader}{Problem \ProblemNumber}	% Don't change this

\begin{document}

\vspace{\baselineskip}	% Add some vertical space
\textbf{I collaborated with:} 


\vspace{\baselineskip}	% Add some vertical space
\textbf{Solution}\\
Claim: Any instance of Vertex Cover can be solved using Monotone Satisfiability with Few True Variables.
\begin{proof}
	Monotone Satisfiability with Few True Variables problem is in NP since any assignment of values to variables can be verified in polynomial time. We show that Monotone Satisfiability with Few True Variables is NP-complete by showing a reduction from Vertex Cover to Monotone Satisfiability with Few True Variables.\\
	Suppose we have a graph $G=(E,V)$ and a number $k$. We want to find a vertex cover of $G$ of size at most $k$. We can find such cover by converting this problem to an equivalent Monotone Satisfiability with Few True Variables problem. We have variable $x_{i}$ for $v_{i}\in V$. We also have clause $C_{j}=(x_{u}\lor x_{v})$ for each edge $e_{j} = (v_{u},v_{v})$. All in all, we have clauses for every edge and variables for every vertex in $G$, and we want to know if all clauses can be satisfied by setting at most $k$ variables to 1.\\
	We claim that a solution of Vertex cover is "yes" iff the answer to Monotone Satisfiability with Few True Variables is also "yes." Suppose we have a vertex cover $S$ of size at most $k$, where all variables corresponding to vertices in $S$ are set to 1. Since clauses are defined by edges, each clause contains at least one variable set to one; so, all clauses are satisfied. Conversely, if we have a solution to the Monotone Satisfiability with Few True Variable problem laid out, then each clause contains a variable set to 1, totaling up to $k$ vertices set to 1. These vertices are incident to every edge because they appear in all clauses, so the set of all these vertices correspond to a vertex cover of $G$ of size at most $k$. 
\end{proof}
%End of feedback section

% DO NOT DELETE ANYTHING BELOW THIS LINE
\end{document}
