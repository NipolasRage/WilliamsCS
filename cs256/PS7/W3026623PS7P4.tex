\documentclass[10pt]{article}

% DO NOT EDIT THE LINES BETWEEN THE TWO LONG HORIZONTAL LINES

%---------------------------------------------------------------------------------------------------------

% Packages add extra functionality.
\usepackage{times,graphicx,epstopdf,fancyhdr,amsfonts,amsthm,amsmath,algorithm,xspace,hyperref}
\usepackage[left=1in,top=1in,right=1in,bottom=1in]{geometry}
\usepackage{sect sty}	%For centering section headings
\usepackage{enumerate}	%Allows more labeling options for enumerate environments 
\usepackage{epsfig}
\usepackage[space]{grffile}
\usepackage{algpseudocode}

% This will set LaTeX to look for figures in the same directory as the .tex file
\graphicspath{.} %The dot means current directory.

\pagestyle{fancy}

\lhead{\WilliamsID}
\chead{Problem Set \PSNumber \ --- Problem \ProblemNumber}
\rhead{\today}
\lfoot{CSci 256: Algorithm Design}
\cfoot{\thepage}
\rfoot{Spring 2018}

% Some commands for changing header and footer format
\renewcommand{\headrulewidth}{0.4pt}
\renewcommand{\headwidth}{\textwidth}
\renewcommand{\footrulewidth}{0.4pt}

% These let you use common environments
\newtheorem{claim}{Claim}
\newtheorem{definition}{Definition}
\newtheorem{theorem}{Theorem}
\newtheorem{lemma}{Lemma}
\newtheorem{observation}{Observation}
\newtheorem{question}{Question}

\setlength{\parindent}{0cm}
\usepackage{tikz}
\usepackage{enumitem}
%---------------------------------------------------------------------------------------------------------

% DON'T CHANGE ANYTHING ABOVE HERE

% Edit below as instructed
\newcommand{\WilliamsID}{W3026623}	% Put you Williams ID in the braces
\newcommand{\PSNumber}{7}			% Put the problem set # in the braces
\newcommand{\ProblemNumber}{4}		% Put the problem # in the braces
\newcommand{\ProblemHeader}{Problem \ProblemNumber}	% Don't change this

\begin{document}

\vspace{\baselineskip}	% Add some vertical space
\textbf{I collaborated with:} 


\vspace{\baselineskip}	% Add some vertical space
\textbf{Solution}\\
Claim: Any instance of Subset Sum can be solved using Zero-Weight Cycles.
\begin{proof}
Zero-Weight Cycles is in NP since a simple cycle can be verified to be a zero-weight cycle by adding the edges connecting the vertices. We show that Zero-Weight Cycles is NP-complete by showing a reduction from Subset Sum to Zero-Weight Cycles.\\
Suppose we have a set $S = \{s_{1},s_{2},...,s_{n}\}$ and a target sum $T$. We want to find a subset $B$ of $S$ such that $\sum_{b\in B}b=T$. We can find such subset by converting this problem to an equivalent Zero-Weight Cycles problem. We create a graph $G=(V,E)$ with $V=\{v_{0},v_{1},...,v_{n}\}$ where $v_{i}$ corresponds to $s_{i}$ and $v_{0}$ is an initial node. We create edges from every $v_{u}$ to every $v_{j}$ where $u<j$ with weight equal to $s_{j}$; and, we also create edges from every $v_{j}$ to $v_{0}$ with weight equal to $-T$. All in all, we have vertices in $G$ that correspond to each element in $S$ along with an extra vertex $v_{0}$ that signals a starting point for our cycle. Furthermore, we have edges going forward from vertices with lower index to vertices with greater index that signify picking the corresponding element of $S$ with the greater index. We also have edges going from every vertex other than $v_{0}$ to $v_{0}$.\\
We claim that a solution of Subset Sum is "yes" iff the answer to Zero-Weight Cycles is also "yes." Suppose we have a subset $B$ of $S$ whose elements add up to $T$. Since vertices correspond to elements of $S$ and weights correspond to the value of the elements, then $B$ shows the path of the Zero-Weight cycle. Starting from $v_{0}$, we go to the vertex that correspond to the next element of $B$. When traveling to a vertex, we add the weight of the edge which happens to be the value of the corresponding element in our subset. When we are done visiting all of the vertices that map to elements of $B$ then our sum is $T$ by definition. The edge back to $v_{0}$ is $-T$, so we have a zero-weight cycle. Conversely, if we have a solution to the Zero-Weight Cycles problem laid out, then all edges traversed are positive except for the one going back to $v_{0}$; this edge has a weight of $-T$. Thus, the sum of all other edges is $T$ and we can collect the vertices they point to to find a subset sum.
\end{proof}
%End of feedback section

% DO NOT DELETE ANYTHING BELOW THIS LINE
\end{document}
