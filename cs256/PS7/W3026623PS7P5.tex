\documentclass[10pt]{article}

% DO NOT EDIT THE LINES BETWEEN THE TWO LONG HORIZONTAL LINES

%---------------------------------------------------------------------------------------------------------

% Packages add extra functionality.
\usepackage{times,graphicx,epstopdf,fancyhdr,amsfonts,amsthm,amsmath,algorithm,xspace,hyperref}
\usepackage[left=1in,top=1in,right=1in,bottom=1in]{geometry}
\usepackage{sect sty}	%For centering section headings
\usepackage{enumerate}	%Allows more labeling options for enumerate environments 
\usepackage{epsfig}
\usepackage[space]{grffile}
\usepackage{algpseudocode}

% This will set LaTeX to look for figures in the same directory as the .tex file
\graphicspath{.} %The dot means current directory.

\pagestyle{fancy}

\lhead{\WilliamsID}
\chead{Problem Set \PSNumber \ --- Problem \ProblemNumber}
\rhead{\today}
\lfoot{CSci 256: Algorithm Design}
\cfoot{\thepage}
\rfoot{Spring 2018}

% Some commands for changing header and footer format
\renewcommand{\headrulewidth}{0.4pt}
\renewcommand{\headwidth}{\textwidth}
\renewcommand{\footrulewidth}{0.4pt}

% These let you use common environments
\newtheorem{claim}{Claim}
\newtheorem{definition}{Definition}
\newtheorem{theorem}{Theorem}
\newtheorem{lemma}{Lemma}
\newtheorem{observation}{Observation}
\newtheorem{question}{Question}

\setlength{\parindent}{0cm}
\usepackage{tikz}
\usepackage{enumitem}
%---------------------------------------------------------------------------------------------------------

% DON'T CHANGE ANYTHING ABOVE HERE

% Edit below as instructed
\newcommand{\WilliamsID}{W3026623}	% Put you Williams ID in the braces
\newcommand{\PSNumber}{7}			% Put the problem set # in the braces
\newcommand{\ProblemNumber}{5}		% Put the problem # in the braces
\newcommand{\ProblemHeader}{Problem \ProblemNumber}	% Don't change this

\begin{document}

\vspace{\baselineskip}	% Add some vertical space
\textbf{I collaborated with:} 


\vspace{\baselineskip}	% Add some vertical space
\textbf{Solution}\\
Claim: Any instance of Hamiltonian Path can be solved using Perfect Assembly.
\begin{proof}
Perfect Assembly is in NP because any assembly of elements of $S$ can be checked to be a perfect assembly with respect to $T$ by walking through the assembly and checking if the last $l$ symbols of an element and the first $l$ symbols of the next element form a string contained in $T$. This can be done in polynomial time proportional to $S$ and $T$ for every two elements of $S$.\\
Suppose we have a directed graph $G=(V,E)$. We want to find a Hamiltonian path in $G$. We can find such path by transforming this graph into a perfect assembly problem. We will label the vertices in a way that the length of the label is even. We label vertices $v_{1}$ to $v_{|V|}$ with numbers $1$ to $|V|$ such that $v_{i}$ is labeled $i$. Before labeling, we want all of the labels to have the same number of digits, so we must fill the highest order digits with zero if they do not exist (if $|V| = 10$ then the vertex with label 3 will become 03). The labels will also be doubles, so a label of 03 becomes 0303. Each vertex now has a label of $2l$ digits. The labels of the vertices becomes our set $S$. Edges in our graph will become elements of $T$. An edge $e=(u,j)$ will be labeled with the last $l$ digits of $u$ and the first $l$ digits of $j$. All in all, we have created a set $S$ where elements go from 1 to $|V|$ such that they have the same number of digits and they are of length $2l$. We also have a set $T$ where each element is a $corroboration$ of two connecting vertex labels.\\
We claim a solution for a Hamiltonian Path is "yes" iff a solution to Perfect Assembly is also "yes." Suppose we have a hamiltonian path $P$. An edge from vertex $u$ to vertex $j$ tells us that an element $t\in T$ corroborates the pair $(s_{u},s_{j})$. Thus, $P$ gives us an assembly of elements such that there exist an elements $t_{k}\in T$ that corroborate adjacent elements. Conversely, a perfect assembly of elements of $S$ gives us a hamiltonian path because elements can only be adjacent if their corresponding vertices in $G$ are also adjacent. This is a path because perfect assemblies cannot repeat elements. Even though perfect assembly solutions can repeat an element $t$, the instance we solve cannot have a repetition of an edge. This is because vertices are not repeated, so edges cannot be repeated.
\end{proof}
%End of feedback section

% DO NOT DELETE ANYTHING BELOW THIS LINE
\end{document}
