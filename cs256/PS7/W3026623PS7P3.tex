\documentclass[10pt]{article}

% DO NOT EDIT THE LINES BETWEEN THE TWO LONG HORIZONTAL LINES

%---------------------------------------------------------------------------------------------------------

% Packages add extra functionality.
\usepackage{times,graphicx,epstopdf,fancyhdr,amsfonts,amsthm,amsmath,algorithm,xspace,hyperref}
\usepackage[left=1in,top=1in,right=1in,bottom=1in]{geometry}
\usepackage{sect sty}	%For centering section headings
\usepackage{enumerate}	%Allows more labeling options for enumerate environments 
\usepackage{epsfig}
\usepackage[space]{grffile}
\usepackage{algpseudocode}

% This will set LaTeX to look for figures in the same directory as the .tex file
\graphicspath{.} %The dot means current directory.

\pagestyle{fancy}

\lhead{\WilliamsID}
\chead{Problem Set \PSNumber \ --- Problem \ProblemNumber}
\rhead{\today}
\lfoot{CSci 256: Algorithm Design}
\cfoot{\thepage}
\rfoot{Spring 2018}

% Some commands for changing header and footer format
\renewcommand{\headrulewidth}{0.4pt}
\renewcommand{\headwidth}{\textwidth}
\renewcommand{\footrulewidth}{0.4pt}

% These let you use common environments
\newtheorem{claim}{Claim}
\newtheorem{definition}{Definition}
\newtheorem{theorem}{Theorem}
\newtheorem{lemma}{Lemma}
\newtheorem{observation}{Observation}
\newtheorem{question}{Question}

\setlength{\parindent}{0cm}
\usepackage{tikz}
\usepackage{enumitem}
%---------------------------------------------------------------------------------------------------------

% DON'T CHANGE ANYTHING ABOVE HERE

% Edit below as instructed
\newcommand{\WilliamsID}{W3026623}	% Put you Williams ID in the braces
\newcommand{\PSNumber}{7}			% Put the problem set # in the braces
\newcommand{\ProblemNumber}{3}		% Put the problem # in the braces
\newcommand{\ProblemHeader}{Problem \ProblemNumber}	% Don't change this

\begin{document}

\vspace{\baselineskip}	% Add some vertical space
\textbf{I collaborated with:} 


\vspace{\baselineskip}	% Add some vertical space
\textbf{Solution}\\
Claim: Any instance of Set Packing can be solved using Multiple Interval Scheduling.
\begin{proof}
	Multiple Interval Scheduling is in NP because we can check if jobs overlap given a scheduling of at least $k$ jobs in polynomial time. We show that Multiple Interval Scheduling is NP-complete by showing a reduction from Set Packing to Multiple Interval Scheduling.\\
	Suppose we have a universe $U=\{u_{1},u_{2},...,u_{j}\}$ and disjoint subsets $B_{1},B_{2},...,B_{n}$ of the universe and a number $k$. We can map elements of $U$ to disjoint intervals from 9 A.M to 5 P.M and subsets correspond to jobs taking up some intervals. This takes time proportional to the size of our universe.\\
	We claim that a solution of Set Packing is "yes" iff the answer to Multiple Interval Scheduling is also "yes." Suppose we have a set packing solution $S$ of size at least $k$ subsets. Since the subsets are disjoint, then the jobs they map to do not overlap. Conversely, if we have a solution to the Multiple Interval Scheduling problem laid out, then each job corresponds to a disjoint subset of our universe $U$. There are at least $k$ nonintersecting jobs, so it must be the case that there are at least $k$ disjoint subsets of $U$.
\end{proof}
%End of feedback section

% DO NOT DELETE ANYTHING BELOW THIS LINE
\end{document}
