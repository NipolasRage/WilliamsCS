\documentclass[10pt]{article}

% DO NOT EDIT THE LINES BETWEEN THE TWO LONG HORIZONTAL LINES

%---------------------------------------------------------------------------------------------------------

% Packages add extra functionality.
\usepackage{times,graphicx,epstopdf,fancyhdr,amsfonts,amsthm,amsmath,algorithm,algorithmic,xspace,hyperref}
\usepackage[left=1in,top=1in,right=1in,bottom=1in]{geometry}
\usepackage{sect sty}	%For centering section headings
\usepackage{enumerate}	%Allows more labeling options for enumerate environments 
\usepackage{epsfig}
\usepackage[space]{grffile}

% This will set LaTeX to look for figures in the same directory as the .tex file
\graphicspath{.} %The dot means current directory.

\pagestyle{fancy}

\lhead{\WilliamsID}
\chead{Final \ --- Problem \ProblemNumber}
\rhead{\today}
\lfoot{CSci 256: Algorithm Design}
\cfoot{\thepage}
\rfoot{Spring 2018}

% Some commands for changing header and footer format
\renewcommand{\headrulewidth}{0.4pt}
\renewcommand{\headwidth}{\textwidth}
\renewcommand{\footrulewidth}{0.4pt}

% These let you use common environments
\newtheorem{claim}{Claim}
\newtheorem{definition}{Definition}
\newtheorem{theorem}{Theorem}
\newtheorem{lemma}{Lemma}
\newtheorem{observation}{Observation}
\newtheorem{question}{Question}

\setlength{\parindent}{0cm}


%---------------------------------------------------------------------------------------------------------

% DON'T CHANGE ANYTHING ABOVE HERE

% Edit below as instructed
\newcommand{\WilliamsID}{W3026623}	% Put you Williams ID in the braces
\newcommand{\ProblemNumber}{2}		% Put the problem # in the braces
\newcommand{\ProblemHeader}{Problem \ProblemNumber}	% Don't change this

\begin{document}


\textbf{Solution}
\\ This problem is in NP since a claimed set $\mathcal{P}$ of directed paths in $G$ can be checked to have $k$ pairwise vertex-disjoint paths in polynomial time. Since this appears to be a packing problem, let’s try a reduction from Set Packing. An instance of Set Packing is a collection $S_{1}, . . . , S_{m}$ of subsets of a set $S={s_{1}, . . . , s_{n}}$ and an integer $k$; a solution is a collection $S_{i_{1}}, . . . , S_{i_{k}}$ of the sets, no two of which intersect.\\
To reduce this to the problem posted, imagine that we build a strongly connected graph $G = (V,E)$ where $V = S$. The collection of subsets $S_{1}, . . . , S_{m}$ corresponds to a collection of vertices $V_{1}, . . . , V_{m}$, where vertices in the same subset form a directed path. In this reduction a set $\mathcal{P}$ can be found. \\
We claim that a solution to Set Packing is "yes" if and only if a solution to the reduction is also "yes." Suppose we have a Set Packing solution $Y$ of at least $k$ subsets. Since the subsets are disjoint, the path they map to are pairwise vertex-disjoint. Conversely, each path $\mathcal{P}$ corresponds to a subset in an instance of Set Packing. These paths are vertex-disjoint, so the subsets must be disjoint as well.
%End of feedback section

% DO NOT DELETE ANYTHING BELOW THIS LINE
\end{document}
