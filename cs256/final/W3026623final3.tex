\documentclass[10pt]{article}

% DO NOT EDIT THE LINES BETWEEN THE TWO LONG HORIZONTAL LINES

%---------------------------------------------------------------------------------------------------------

% Packages add extra functionality.
\usepackage{times,graphicx,epstopdf,fancyhdr,amsfonts,amsthm,amsmath,algorithm,algorithmic,xspace,hyperref}
\usepackage[left=1in,top=1in,right=1in,bottom=1in]{geometry}
\usepackage{sect sty}	%For centering section headings
\usepackage{enumerate}	%Allows more labeling options for enumerate environments 
\usepackage{epsfig}
\usepackage[space]{grffile}

% This will set LaTeX to look for figures in the same directory as the .tex file
\graphicspath{.} %The dot means current directory.

\pagestyle{fancy}

\lhead{\WilliamsID}
\chead{Final \ --- Problem \ProblemNumber}
\rhead{\today}
\lfoot{CSci 256: Algorithm Design}
\cfoot{\thepage}
\rfoot{Spring 2018}

% Some commands for changing header and footer format
\renewcommand{\headrulewidth}{0.4pt}
\renewcommand{\headwidth}{\textwidth}
\renewcommand{\footrulewidth}{0.4pt}

% These let you use common environments
\newtheorem{claim}{Claim}
\newtheorem{definition}{Definition}
\newtheorem{theorem}{Theorem}
\newtheorem{lemma}{Lemma}
\newtheorem{observation}{Observation}
\newtheorem{question}{Question}

\setlength{\parindent}{0cm}


%---------------------------------------------------------------------------------------------------------

% DON'T CHANGE ANYTHING ABOVE HERE

% Edit below as instructed
\newcommand{\WilliamsID}{W3026623}	% Put you Williams ID in the braces
\newcommand{\ProblemNumber}{3}		% Put the problem # in the braces
\newcommand{\ProblemHeader}{Problem \ProblemNumber}	% Don't change this

\begin{document}


\textbf{Solution}
\\ Let $G = (V,E)$ be our input 2-connected outerplanar graph, $k=2$ denote the starting minimum integer such that $G$ has a proper $k-$coloring. Let the term $odd\ face$ denote a face bounded by an odd number of vertices and is not completely bounded by diagonals. If $|V|=1$ then return $k=1$. If $|V|$ is odd then $k=3$. \begin{itemize}
	\item Now, pick a starting vertex $v$ and walk along outer edges until a diagonal is reached.
	\item Mark diagonal as visited. 
	\item If this diagonal belongs to an odd face then $k=3$.
	\item Continue walking along the outer edges until another unvisited diagonal is reached. 
	\item If the face bounded by this diagonal is an odd face and is adjacent to three other odd faces then $k = k + 1$.
	\item Repeat these last two steps until $k=4$ or until the all of $G$ is visited.
\end{itemize}
Time Complexity: Walking around outer edges $O(|V|+|E|)$ + check if diagonal belongs to an odd face: $O(|V|+|E|)$ * check adjacency of odd faces: $O(|V|+|E|) = O((|V|+|E|)^{2})$\\
Space Complexity: $O(|V|+|E|)$\\
\begin{proof}
	I will now prove that this algorithn produces the minimum amount of colors for a proper $k-$coloring of a graph $G=(V,E)$. The first thing to address is the upper and lower bounds of $k$. If $|V|=1$ then we know there can only be one color in our graph. Other than that, the lowest $k$ a graph could have is 2 if $|V|$ is even. This is because the colors can simply alternate if there are no diagonals. Now, if $|V|$ is odd, then there must be one more color even if there are no diagonals. The algorithm has an upperbound of 4 due to Appel and Haken's Four Color Theorem. Now, suppose the value of $k$ is currently less than 4 for a graph $G$ and we are able to add a diagonal. There are three cases when adding this diagonal.\\
	Case 1: If this diagonal creates an even face then it partitions the graph into two subgraphs with at least one of them having an even amount of vertices. This means that the even subgraph can be colored with 2 colors and the odd subgraph can be colored with the value of $k$.\\
	Case 2: Adding a diagonal but not having three adjacent odd faces. In this case, there exists a face bounded by diagonals. This allows for alternation of $k$ colors without introducing a new one because the diagonals appear in alternating places.\\
	Case 3: Adding a diagonal and having four adjacent odd faces. This will cause a vertex to be incident to three other vertices, so another color would be needed.\\
	We have now proved that this algorithm will produce the minimum number of colors needed for a $k-$coloring of $G$.
\end{proof}
%End of feedback section

% DO NOT DELETE ANYTHING BELOW THIS LINE
\end{document}
