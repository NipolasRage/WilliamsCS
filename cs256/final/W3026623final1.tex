\documentclass[10pt]{article}

% DO NOT EDIT THE LINES BETWEEN THE TWO LONG HORIZONTAL LINES

%---------------------------------------------------------------------------------------------------------

% Packages add extra functionality.
\usepackage{times,graphicx,epstopdf,fancyhdr,amsfonts,amsthm,amsmath,algorithm,algorithmic,xspace,hyperref}
\usepackage[left=1in,top=1in,right=1in,bottom=1in]{geometry}
\usepackage{sect sty}	%For centering section headings
\usepackage{enumerate}	%Allows more labeling options for enumerate environments 
\usepackage{epsfig}
\usepackage[space]{grffile}

% This will set LaTeX to look for figures in the same directory as the .tex file
\graphicspath{.} %The dot means current directory.

\pagestyle{fancy}

\lhead{\WilliamsID}
\chead{Final \ --- Problem \ProblemNumber}
\rhead{\today}
\lfoot{CSci 256: Algorithm Design}
\cfoot{\thepage}
\rfoot{Spring 2018}

% Some commands for changing header and footer format
\renewcommand{\headrulewidth}{0.4pt}
\renewcommand{\headwidth}{\textwidth}
\renewcommand{\footrulewidth}{0.4pt}

% These let you use common environments
\newtheorem{claim}{Claim}
\newtheorem{definition}{Definition}
\newtheorem{theorem}{Theorem}
\newtheorem{lemma}{Lemma}
\newtheorem{observation}{Observation}
\newtheorem{question}{Question}

\setlength{\parindent}{0cm}


%---------------------------------------------------------------------------------------------------------

% DON'T CHANGE ANYTHING ABOVE HERE

% Edit below as instructed
\newcommand{\WilliamsID}{W3026623}	% Put you Williams ID in the braces
\newcommand{\ProblemNumber}{1}		% Put the problem # in the braces
\newcommand{\ProblemHeader}{Problem \ProblemNumber}	% Don't change this

\begin{document}


\textbf{Solution}
\\ a) If there are more clubs than students, then no such set $T$ exists because no individual person can be a treasurer for more than one group by the pigeonhole principle. 
\\First make a bipartide graph $G=(S,Y,E)$ where $S=\{x_{1},...,x_{k}\}$ is the set of students part of a club, $Y$ is a collection of subsets of $S$ denoting clubs and where elements of $Y_{i}$ are students that are members of the club, and $E=\{(x,Y_{i}):x\in Y_{i}\}$, meaning that an edge is made from a student in $S$ to a club in $Y$ if the student is part of the club. A successful set of tresurers corresponds to a set $T=\{s_{1},...,s_{n}\} \subseteq S$ such that each $s_{i}\in Y_{i}$ and there exists only one tresurer per group.\\
To determine whether such a set $T$ exists, transform $G$ into a flow network $G'$ by:
\begin{enumerate}
	\item Directing all edges of $G$ from $S$ to $Y$.
	\item Adding a vertex $s$ with edges to all $x\in S$, and a vertex $t$ with edges from all $y\in Y$ to $t$.
	\item Giving every edge in $G'$ capacity 1.
\end{enumerate}
Claim: The $n$ clubs can each get a treasurer if and only if $G'$ has a flow of value $n$.
\begin{proof}
	$\Rightarrow$ Assume the set of treasurers exists and let $A$ be the set of edges representing the assignment of students to the club they are treasuring.  Let $f$ be a flow on $G'$ with all edges having flow 0.  Augment $f$ as follows:  For each edge $\{x, y\} \in A$,  add 1 to the flow of the directed edges $(s, x),(x, y)$, and $(y, t)$ in $G'$. Note that doing so preserves flow conservation, and increases the value of the flow $f$ by 1. At the end of this process, no edge has flow exceeding capacity since no edges ever receive more than 1 unit of flow. Moreover, a student in $S$ receiving a flow of 1 must not treasure another club because of flow conservation. Also, a student being treasurer means that they have an edge in $A$. The value of this flow is $n$ since the set of treasurers exists so $|A|=n$, meaning that every club received a unit of flow. Thus, all edges from $Y$ to $t$ have 1 unit of flow.\\
	$\Leftarrow$ Assume there is an integer flow $f$ of value $n$. Then each edge $(y, t)$ must have 1 unit of flow, and so each club $y$ must have an incoming edge (from some $x\in S$ with 1 unit of flow). Each $x\in S$ can have at most 1 flow leaving it since edges $(s,x)$ have a capacity of 1. Thus, the flow of $n$ from clubs to the sink correspond to $n$ students being treasurers of said club. 
\end{proof}
Space Complexity: $|S||Y|+2$ vertices * at most every student is is every club $|S||Y|+n+|S| = O((|S||Y|)^{2})$\\
Time Complexity: $O(|V||E|)=O((|S||Y|)^{2})$\\
b) The bipartide graph $G$ we created might not always have a perfect matching. A certificate we can use to verify that there does not exists a perfect matching is one similar to Hall's Theorem, that is that for some $X \subseteq S$, $|\Gamma(X)| < |X|$. This means that for some subsets of $S$, there are not enough conncetions to $Y$. Assume there is a sub-collection $Y_{i_{1}},...,Y_{i_{k}}$ such that $|Y_{i_{1}},...,Y_{i_{k}}|<k$. This means that in this sub-collection of groups, there are less members than groups. In other words, some students in $S$ do not have sufficient edges to $Y$. In fact, this means that a perfect matching cannot be made from $S$ to $Y$. Conversely, if no such a set $T \subseteq S$ exists, then some students do not have sufficient edges to a set of groups. Meaning that some groups are short of students.

%End of feedback section

% DO NOT DELETE ANYTHING BELOW THIS LINE
\end{document}
