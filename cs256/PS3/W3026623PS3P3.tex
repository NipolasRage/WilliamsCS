\documentclass[10pt]{article}

% DO NOT EDIT THE LINES BETWEEN THE TWO LONG HORIZONTAL LINES

%---------------------------------------------------------------------------------------------------------

% Packages add extra functionality.
\usepackage{times,graphicx,epstopdf,fancyhdr,amsfonts,amsthm,amsmath,algorithm,algorithmic,xspace,hyperref}
\usepackage[left=1in,top=1in,right=1in,bottom=1in]{geometry}
\usepackage{sect sty}	%For centering section headings
\usepackage{enumerate}	%Allows more labeling options for enumerate environments 
\usepackage{epsfig}
\usepackage[space]{grffile}

% This will set LaTeX to look for figures in the same directory as the .tex file
\graphicspath{.} %The dot means current directory.

\pagestyle{fancy}

\lhead{\WilliamsID}
\chead{Problem Set \PSNumber \ --- Problem \ProblemNumber}
\rhead{\today}
\lfoot{CSci 256: Algorithm Design}
\cfoot{\thepage}
\rfoot{Spring 2018}

% Some commands for changing header and footer format
\renewcommand{\headrulewidth}{0.4pt}
\renewcommand{\headwidth}{\textwidth}
\renewcommand{\footrulewidth}{0.4pt}

% These let you use common environments
\newtheorem{claim}{Claim}
\newtheorem{definition}{Definition}
\newtheorem{theorem}{Theorem}
\newtheorem{lemma}{Lemma}
\newtheorem{observation}{Observation}
\newtheorem{question}{Question}

\setlength{\parindent}{0cm}


%---------------------------------------------------------------------------------------------------------

% DON'T CHANGE ANYTHING ABOVE HERE

% Edit below as instructed
\newcommand{\WilliamsID}{W3026623}	% Put you Williams ID in the braces
\newcommand{\PSNumber}{3}			% Put the problem set # in the braces
\newcommand{\ProblemNumber}{3}		% Put the problem # in the braces
\newcommand{\ProblemHeader}{Problem \ProblemNumber}	% Don't change this

\begin{document}

\vspace{\baselineskip}	% Add some vertical space
\textbf{I collaborated with:} 


\vspace{\baselineskip}	% Add some vertical space
\textbf{Problem}

A {\em feedback edge set} of a graph $G$ is a subset $F$ of the edges such that every cycle in $G$ contains at least one edge in $F$. In other words, removing every edge in $F$ makes the graph $G$ acyclic.   Describe and analyze a fast algorithm to compute the minimum weight feedback edge set of  a given edge-weighted graph.  Hint.  Relate this problem to some kind of spanning tree problem.
  
\textbf{Solution}
\begin{algorithm*}[h] 
	\caption{Kruskal's adaptation} 
	\begin{algorithmic}[1] 
		\STATE$T\leftarrow (V,\emptyset)$ // Eventual maximum cost spanning tree
		\STATE $R\leftarrow E$
		\STATE Let $F$ be an empty set
		\STATE $MakeUnionFind(V)$
		\WHILE {$|E (T )| < |V |-1$}
		\STATE Remove heaviest edge $e = \{u,v\} \in R$ from $R$
		\STATE $uName = Find(u); vName = Find(v)$
		\IF {$uName \ne vName$}
		\STATE Add $e$ to $T$
		\STATE $Union(uName, vName)$
		\ELSE 
		\STATE Add $e$ to $F$
		\ENDIF
		\ENDWHILE
		\RETURN $F$
	\end{algorithmic} 
\end{algorithm*}\\
The runtime of this algorithm is the same as Kruskal's as the extra step takes constant time: $O(|V|+|E|)*O(1) = O(|V|+|E|)$\\
This algorithm is adding the maximum spanning edges of the graph to $T$. We throw away edges that would create a cycle. In the problem we looked at in class, this edge that we threw away was the maximum edge of the cycle it would create. If we reverse the algorithm to make a maximum cost spanning tree, we put the minimum edges that will create cycles into $F$.


%End of feedback section

% DO NOT DELETE ANYTHING BELOW THIS LINE
\end{document}
