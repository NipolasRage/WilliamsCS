\documentclass[10pt]{article}

% DO NOT EDIT THE LINES BETWEEN THE TWO LONG HORIZONTAL LINES

%---------------------------------------------------------------------------------------------------------

% Packages add extra functionality.
\usepackage{times,graphicx,epstopdf,fancyhdr,amsfonts,amsthm,amsmath,algorithm,algorithmic,xspace,hyperref}
\usepackage[left=1in,top=1in,right=1in,bottom=1in]{geometry}
\usepackage{sect sty}	%For centering section headings
\usepackage{enumerate}	%Allows more labeling options for enumerate environments 
\usepackage{epsfig}
\usepackage[space]{grffile}

% This will set LaTeX to look for figures in the same directory as the .tex file
\graphicspath{.} %The dot means current directory.

\pagestyle{fancy}

\lhead{\WilliamsID}
\chead{Problem Set \PSNumber \ --- Problem \ProblemNumber}
\rhead{\today}
\lfoot{CSci 256: Algorithm Design}
\cfoot{\thepage}
\rfoot{Spring 2018}

% Some commands for changing header and footer format
\renewcommand{\headrulewidth}{0.4pt}
\renewcommand{\headwidth}{\textwidth}
\renewcommand{\footrulewidth}{0.4pt}

% These let you use common environments
\newtheorem{claim}{Claim}
\newtheorem{definition}{Definition}
\newtheorem{theorem}{Theorem}
\newtheorem{lemma}{Lemma}
\newtheorem{observation}{Observation}
\newtheorem{question}{Question}

\setlength{\parindent}{0cm}


%---------------------------------------------------------------------------------------------------------

% DON'T CHANGE ANYTHING ABOVE HERE

% Edit below as instructed
\newcommand{\WilliamsID}{W3026623}	% Put you Williams ID in the braces
\newcommand{\PSNumber}{3}			% Put the problem set # in the braces
\newcommand{\ProblemNumber}{1}		% Put the problem # in the braces
\newcommand{\ProblemHeader}{Problem \ProblemNumber}	% Don't change this

\begin{document}

\vspace{\baselineskip}	% Add some vertical space
\textbf{I collaborated with:} W3023339


\vspace{\baselineskip}	% Add some vertical space
\textbf{Problem}

Let $\Sigma$ be a finite alphabet (evocative synonym for 'set') and let $w = \tau_1 \tau_2 \ldots  \tau_m$ and $p = \sigma_1 \sigma_2 \ldots \sigma_n$
be two finite sequences of elements of $\Sigma$.\footnote{Tradition: Greek letters are used when denoting alphabets and their ``letters".}  We say that the word $w$ {\em contains} the pattern $p$ if $\sigma_1, \sigma_2,\ldots,  \sigma_n$
occur in $w$ in the same order (but not necessarily consecutively) that they occur in $p$.   That is, $w$ contains $p$ if for some
$\tau_{i_1}, \ldots , \tau_{i_n}$ with $i_1 < i_2 < \ldots < i_n$ we have $\tau_{i_1} = \sigma_1, \ldots , \tau_{i_n} = \sigma_n$.
Design an $O(n+m)$ (greedy) algorithm to determine, given a pair $w$ and $p$, whether $w$ contains the pattern $p$, and prove that your
algorithm correctly solves the problem in this amount of time.
  
\textbf{Solution}
\begin{algorithm*}[h] 
	\caption{Determines whether $w$ contains the pattern $p$} 
	\begin{algorithmic}[1] 
		\REQUIRE sequences $w$ and $p$ in $\Sigma$.
		\medskip 
		\STATE $index \leftarrow 0$
		\FOR {every $\tau_{i}$ in $w$}
		\IF {$\tau_{i} = \sigma_{index}$}
		\STATE $index \leftarrow index + 1$ 
		\ENDIF
		\ENDFOR
		\RETURN true if $index = n+1$ else false
	\end{algorithmic} 
\end{algorithm*}

\begin{proof}
	Base Case: Let $p$ be of length 1. As long as $\sigma \in w$, then $w$ contains $p$. Our algorithm will walk through $w$, comparing every letter to $\sigma$ and, if a match is found, $index$ will be 1, so true will be returned.\\ 
	Inductive Hypothesis: Our algorithm works for a pattern $p$ of size $k$, where $k$ is an integer. That is, the algorithm returns true if a word $w$ contains $p$.\\
	Inductive Step: We must show our algorithm works for a pattern of size $k+1$. From our inductive hypothesis, we know it will check if the first $k$ elements of $p$ are in $w$, so we have to see what happens with this extra element. If $w$ is of size less than $k+1$, then our $w$ does not contain $p$ and $index$ from our algorithm will not get to be $k+2$. If $w$ is of size greater than $k$, then it already checked the first $k$ elements of $p$ if there were $k$ matches, and $index$ is $k+1$. It will then check for an element $\tau$ that matches this $k+1$ element. If there is a match, $w$ contains $p$ and the algorithm returns true, otherwise $w$ does not contain $p$ and the algorithm returns false.
\end{proof}
%End of feedback section

% DO NOT DELETE ANYTHING BELOW THIS LINE
\end{document}
