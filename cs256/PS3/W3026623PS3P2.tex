\documentclass[10pt]{article}

% DO NOT EDIT THE LINES BETWEEN THE TWO LONG HORIZONTAL LINES

%---------------------------------------------------------------------------------------------------------

% Packages add extra functionality.
\usepackage{times,graphicx,epstopdf,fancyhdr,amsfonts,amsthm,amsmath,algorithm,algorithmic,xspace,hyperref}
\usepackage[left=1in,top=1in,right=1in,bottom=1in]{geometry}
\usepackage{sect sty}	%For centering section headings
\usepackage{enumerate}	%Allows more labeling options for enumerate environments 
\usepackage{epsfig}
\usepackage[space]{grffile}

% This will set LaTeX to look for figures in the same directory as the .tex file
\graphicspath{.} %The dot means current directory.

\pagestyle{fancy}

\lhead{\WilliamsID}
\chead{Problem Set \PSNumber \ --- Problem \ProblemNumber}
\rhead{\today}
\lfoot{CSci 256: Algorithm Design}
\cfoot{\thepage}
\rfoot{Spring 2018}

% Some commands for changing header and footer format
\renewcommand{\headrulewidth}{0.4pt}
\renewcommand{\headwidth}{\textwidth}
\renewcommand{\footrulewidth}{0.4pt}

% These let you use common environments
\newtheorem{claim}{Claim}
\newtheorem{definition}{Definition}
\newtheorem{theorem}{Theorem}
\newtheorem{lemma}{Lemma}
\newtheorem{observation}{Observation}
\newtheorem{question}{Question}

\setlength{\parindent}{0cm}


%---------------------------------------------------------------------------------------------------------

% DON'T CHANGE ANYTHING ABOVE HERE

% Edit below as instructed
\newcommand{\WilliamsID}{W3026623}	% Put you Williams ID in the braces
\newcommand{\PSNumber}{3}			% Put the problem set # in the braces
\newcommand{\ProblemNumber}{2}		% Put the problem # in the braces
\newcommand{\ProblemHeader}{Problem \ProblemNumber}	% Don't change this

\begin{document}

\vspace{\baselineskip}	% Add some vertical space
\textbf{I collaborated with:} 


\vspace{\baselineskip}	% Add some vertical space
\textbf{Problem}

Please answer Question 17 from Chapter 4 of your text.  Try to shoot for an $O(n^2)$ algorithm; it's a bit of a challenge to beat that (although if you're feeling really ambitious, $O(n \log n)$ time is possible...).
  
\textbf{Solution}
\begin{algorithm*}[h] 
	\caption{Determines whether $w$ contains the pattern $p$} 
	\begin{algorithmic}[1] 
		\FOR {every interval $i$}
		\STATE run Single Source Scheduling algorithm from class starting on $i$
		\STATE keep track of the largest solution out of the Single Source Scheduling runs
		\ENDFOR
	\end{algorithmic} 
\end{algorithm*}\\
This algorithm runs in $O(n^{2})$ as there are $n$ intervals and the Single Source Scheduling algorithm takes $O(n)$. $O(n)*O(n)=O(n^{2})$\\
Space: Single Source Scheduling creates $n$ solution that take $O(n)$ space so it takes $O(n^{2})$.\\
Suppose there is an optimal solution $S$ consisting of the maximum set of intervals. $S$ starts with interval $i$, and the Single Source Scheduling algorithm will eventually start with $i$ and determine it is the largest solution. Thus, the output of our solution will be at least as large as $S$.


%End of feedback section

% DO NOT DELETE ANYTHING BELOW THIS LINE
\end{document}
