\documentclass[10pt]{article}

% DO NOT EDIT THE LINES BETWEEN THE TWO LONG HORIZONTAL LINES

%---------------------------------------------------------------------------------------------------------

% Packages add extra functionality.
\usepackage{times,graphicx,epstopdf,fancyhdr,amsfonts,amsthm,amsmath,algorithm,algorithmic,xspace,hyperref}
\usepackage[left=1in,top=1in,right=1in,bottom=1in]{geometry}
\usepackage{sect sty}	%For centering section headings
\usepackage{enumerate}	%Allows more labeling options for enumerate environments 
\usepackage{epsfig}
\usepackage[space]{grffile}
\usepackage{enumitem}

% This will set LaTeX to look for figures in the same directory as the .tex file
\graphicspath{.} %The dot means current directory.

\pagestyle{fancy}

\lhead{\WilliamsID}
\chead{Problem Set \PSNumber \ --- Problem \ProblemNumber}
\rhead{\today}
\lfoot{CSci 256: Algorithm Design}
\cfoot{\thepage}
\rfoot{Spring 2018}

% Some commands for changing header and footer format
\renewcommand{\headrulewidth}{0.4pt}
\renewcommand{\headwidth}{\textwidth}
\renewcommand{\footrulewidth}{0.4pt}

% These let you use common environments
\newtheorem{claim}{Claim}
\newtheorem{definition}{Definition}
\newtheorem{theorem}{Theorem}
\newtheorem{lemma}{Lemma}
\newtheorem{observation}{Observation}
\newtheorem{question}{Question}

\setlength{\parindent}{0cm}


%---------------------------------------------------------------------------------------------------------

% DON'T CHANGE ANYTHING ABOVE HERE

% Edit below as instructed
\newcommand{\WilliamsID}{W3026623}	% Put you Williams ID in the braces
\newcommand{\PSNumber}{3}			% Put the problem set # in the braces
\newcommand{\ProblemNumber}{4}		% Put the problem # in the braces
\newcommand{\ProblemHeader}{Problem \ProblemNumber}	% Don't change this

\begin{document}

\vspace{\baselineskip}	% Add some vertical space
\textbf{I collaborated with:} 


\vspace{\baselineskip}	% Add some vertical space
\textbf{Problem}

In many situations, you might find yourself dealing with a dynamically changing data structure.  One such simple example is that of
dynamically changing edge weights in a graph.  This problem asks you to consider how you would maintain a minimum-cost spanning tree in
such an environment.

Suppose you are given a graph $G$ with weighted edges and a minimum spanning tree $T$ of $G$. 
\begin{description}
	\item [(a)] Describe and analyze an algorithm to update the minimum spanning tree when the weight of a single edge $e$ is decreased.
	\item [(b)] Describe and analyze an  algorithm to update the minimum spanning tree when the weight of a single edge $e$ is increased.
\end{description}
In both cases, the input to your algorithm is the edge $e$ and its new weight; your algorithms should modify $T$ so that it is still a minimum spanning tree.  {\em Hint:  consider $e \in T$ and $e \not\in T$ separately}.
  
\textbf{Solution}
\begin{enumerate}[label=\Alph*]
	\item This algorithm will perform a BFS on the tree until we find one of two things. The edge $e = \{u,v\}\in T$  or $e\not\in T$. If $e$ is in the graph then we just update its weight and we are done. Otherwise, we add $e$ to $T$, which adds a cycle, perform another BFS from $v$, marking every node we encounter. If we encountered a node $b$ twice, then we remove the edge of highest weight in this cycle. We now have our updated MCST.\\
	Runtime: $1^{st}$ BFS $O(|V| + |E|)$ * $2^{nd}$ BFS $O(|V| + |E|)$ * walk through cycle $O(|V| + |E|)= O(3(|V| + |E|)) = O(|V| + |E|)$ 
	\begin{proof}
		Performing our first BFS to find the node of destination will tell us where to change $T$. Now we have two cases, $e \in T$ and $e \not\in T$:\\
		Case $e \in T$: If this is the case then $e$ will be updated to have a lower weight and $T$ mantains its MCST property.\\
		Case $e \not\in T$: In this case, let $e$ connect vertices $u$ to $v$ in $G$. By performing a BFS and finding $v$ we find the potential spot for $e$ and add it to our tree. This forms a cycle and we want to get rid of the edge that costs the most in this cycle. Removing it will end the cycle and reduce the cost of $T$ so that it keeps its MCST property.
	\end{proof}
	\item This algorithm will perform a BFS much like in part $A$ to find one of two things.  The edge $e = \{u,v\}\in T$  or if $e\not\in T$. If $e\not\in T$ then we can increase its weight and be done. If $e \in T$, then we remove it. This disconnects $T$ into two parts, $V$ and $V'$. We perform a BFS from $v$ and mark all of the vertices we visit. Note that these vertices are in $V'$ and not in $V$. We now take a look at our list of edges, $G(E)$. These edges either connect two marked nodes, two unmarked nodes, or one marked node and an unmarked node. We find the lowest cost edge that connects a marked node and an unmarked node and add it to $T$. $T$ is now a MCST.\\
	Runtime: $1^{st}$ BFS $O(|V| + |E|)$ * $2^{nd}$ BFS $O(|V| + |E|)$ * find min edge $O(|E|)$ * add lowest cost edge to $T\ O(|V| + |E|) = O(3|V| + 4|E|) = O(|V| + |E|)$ \\
	\begin{proof}
		Performing the first BFS on $T$ will tell us where node $e$ has the potential to be. We have two cases $e \in T$ and $e \not\in T$:\\
		Case $e \not\in T$: If this is the case then $e$ was not originally in our MCST, so increasing the cost will not influence $T$.\\
		Case $e \in T$: In this case, we find $e$ by looking at the vertices it connects. We update its weight and remove it which disconnects $T$, as there are no cycles in trees. We now have two sets of vertices, $V$ and $V'$. By marking the vertices of $V'$, we can detect the edges that are in this cut. A set of cut edges are those edges that if removed from $G$, they disconnect the graph into two distinct sets of vertices. By finding the smallest cut edge, we are able to connect $T$ with the smallest edge possible. This maintains $T$'s MCST property. 
	\end{proof}
\end{enumerate}


%End of feedback section

% DO NOT DELETE ANYTHING BELOW THIS LINE
\end{document}
