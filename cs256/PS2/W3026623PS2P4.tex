\documentclass[10pt]{article}

% DO NOT EDIT THE LINES BETWEEN THE TWO LONG HORIZONTAL LINES

%---------------------------------------------------------------------------------------------------------

% Packages add extra functionality.
\usepackage{times,graphicx,epstopdf,fancyhdr,amsfonts,amsthm,amsmath,algorithm,xspace,hyperref}
\usepackage[left=1in,top=1in,right=1in,bottom=1in]{geometry}
\usepackage{sect sty}	%For centering section headings
\usepackage{enumerate}	%Allows more labeling options for enumerate environments 
\usepackage{epsfig}
\usepackage[space]{grffile}
\usepackage{algpseudocode}

% This will set LaTeX to look for figures in the same directory as the .tex file
\graphicspath{.} %The dot means current directory.

\pagestyle{fancy}

\lhead{\WilliamsID}
\chead{Problem Set \PSNumber \ --- Problem \ProblemNumber}
\rhead{\today}
\lfoot{CSci 256: Algorithm Design}
\cfoot{\thepage}
\rfoot{Spring 2018}

% Some commands for changing header and footer format
\renewcommand{\headrulewidth}{0.4pt}
\renewcommand{\headwidth}{\textwidth}
\renewcommand{\footrulewidth}{0.4pt}

% These let you use common environments
\newtheorem{claim}{Claim}
\newtheorem{definition}{Definition}
\newtheorem{theorem}{Theorem}
\newtheorem{lemma}{Lemma}
\newtheorem{observation}{Observation}
\newtheorem{question}{Question}

\setlength{\parindent}{0cm}


%---------------------------------------------------------------------------------------------------------

% DON'T CHANGE ANYTHING ABOVE HERE

% Edit below as instructed
\newcommand{\WilliamsID}{W3026623}	% Put you Williams ID in the braces
\newcommand{\PSNumber}{2}			% Put the problem set # in the braces
\newcommand{\ProblemNumber}{4}		% Put the problem # in the braces
\newcommand{\ProblemHeader}{Problem \ProblemNumber}	% Don't change this

\begin{document}

\vspace{\baselineskip}	% Add some vertical space
\textbf{I collaborated with:} W3029343


\vspace{\baselineskip}	% Add some vertical space
\textbf{Problem}

Answer Question 11 from Chapter Three of your text {\em Algorithm Design}.  In addition to designing the algorithm,
justify its correctness and time and space requirements.\\
\textbf{Solution}
\begin{algorithm*}[h] 
	\caption{determines whether a virus introduced at computer $a$ at time $x$ could have infected computer $b$ by time $y$} 
	\begin{algorithmic}
		\Function{Detection}{$C_{a}$,$x$,$C_{b}$,$y$, $m$}
		\State let $G$ be an empty directed graph
		\For{all triplets $(C_{i},C_{j},t_{k})$ in $m$}
		\State $i \leftarrow$ node $(C_{i},t_{k})$ 
		\State $j \leftarrow$ node $(C_{j},t_{k})$ 
		\State add \{$i,j$\} and $\{j,i\}$ to $G$
		\If{we have seen $C_{i}$ or $C_{j}$ before}
		\State add $\{(C_{i},t_{o}),(C_{i},t_{k})\}$ and/or $\{(C_{j},t_{o}),(C_{j},t_{k})\}$ to $G$ where $o$ is the previous time
		\EndIf
		\EndFor
		\State Travel to neighboring node reachable from $C_{a},t$ where $t\ge x$
		\State run $BFS$ from this node
		\If {we reach a node of the form $(C_{b},t)$ where $t \le y$}
		\State\Return $True$
		\EndIf
		\State\Return $False$
		\EndFunction
	\end{algorithmic}
\end{algorithm*} \\
In this algortithm, we build a graph $G$ that keeps track of the flow of the virus since every edge in $G$ points forward in time. We start looking through the graph once $C_{a}$ connects to another computer $C_{i}$. We also keep track of the computers $C_{i}$ interacts with and which computers those interact with. Eventually, a computer that has an edge between a $C_{j}, t_{j}$ and $C_{b},t_{j}$ where $x \le t_{j} \le y$. Thus, we will see if $C_{b}$ gets infected.\\
Time requirement: Adding a node to $G$: $O(m)$ + BFS: $O(m)$ = $O(2m)$.\\
Space requirement: $G$: $O(m)$
%End of feedback section

% DO NOT DELETE ANYTHING BELOW THIS LINE
\end{document}
