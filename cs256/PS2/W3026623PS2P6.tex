\documentclass[10pt]{article}

% DO NOT EDIT THE LINES BETWEEN THE TWO LONG HORIZONTAL LINES

%---------------------------------------------------------------------------------------------------------

% Packages add extra functionality.
\usepackage{times,graphicx,epstopdf,fancyhdr,amsfonts,amsthm,amsmath,algorithm,algorithmic, xspace,hyperref}
\usepackage[left=1in,top=1in,right=1in,bottom=1in]{geometry}
\usepackage{sect sty}	%For centering section headings
\usepackage{enumerate}	%Allows more labeling options for enumerate environments 
\usepackage{epsfig}
\usepackage[space]{grffile}


% This will set LaTeX to look for figures in the same directory as the .tex file
\graphicspath{.} %The dot means current directory.

\pagestyle{fancy}

\lhead{\WilliamsID}
\chead{Problem Set \PSNumber \ --- Problem \ProblemNumber}
\rhead{\today}
\lfoot{CSci 256: Algorithm Design}
\cfoot{\thepage}
\rfoot{Spring 2018}

% Some commands for changing header and footer format
\renewcommand{\headrulewidth}{0.4pt}
\renewcommand{\headwidth}{\textwidth}
\renewcommand{\footrulewidth}{0.4pt}

% These let you use common environments
\newtheorem{claim}{Claim}
\newtheorem{definition}{Definition}
\newtheorem{theorem}{Theorem}
\newtheorem{lemma}{Lemma}
\newtheorem{observation}{Observation}
\newtheorem{question}{Question}

\setlength{\parindent}{0cm}


%---------------------------------------------------------------------------------------------------------

% DON'T CHANGE ANYTHING ABOVE HERE

% Edit below as instructed
\newcommand{\WilliamsID}{W3026623}	% Put you Williams ID in the braces
\newcommand{\PSNumber}{2}			% Put the problem set # in the braces
\newcommand{\ProblemNumber}{6}		% Put the problem # in the braces
\newcommand{\ProblemHeader}{Problem \ProblemNumber}	% Don't change this

\begin{document}

\vspace{\baselineskip}	% Add some vertical space


\vspace{\baselineskip}	% Add some vertical space
\textbf{Problem}\\
There's another well-known algorithm to compute the diameter of a tree that may surprise you.  Here it is
\begin{itemize}
	\item Pick any vertex $v$ in $T$
	\item Find a vertex $u$ that maximizes $dist(v,u)$, using a traversal, for example
	\item Find a vertex $w$ that maximizes $dist(u,w)$
	\item Return $dist(u,w)$ as the diameter of $T$.
\end{itemize}
Prove that this algorithm is correct.
Suggestion: Draw some pictures. Consider the sub-tree $T'$ that consists of the (unique!) paths from $v$ to $u$ and $u$ to $w$.
Imagine some longer path exists---how does it connect to $T'$?\\
\textbf{Solution}\\
\begin{proof}
	The algorithm finds a vertex $u$ that is farthest away from $v$, leaves are the only vertices that can be farthest away from another vertice so $u$ and $w$ are leaves. Consequently, any other leaf is not further away from $v$ or from $u$. Now take a subtree $T'$ that only contains paths from $v$ to $u$ and $u$ to $w$. If there was a longer path containing $u$, then our algorithm would have nade that other leaf $w$. Also, if there was a longer path from $v$, then our algorithm would have nade that other leaf $u$. Suppose there was a longer path from $u$ to $w$. This would mean that there are two paths from $u$ to $w$, which means that $T$ has a cycle. $T$ is a tree so it cannot have cycles. We have reached a contradiction, so there is no longer path from $u$ to $w$. 
\end{proof}
%End of feedback section

% DO NOT DELETE ANYTHING BELOW THIS LINE
\end{document}
