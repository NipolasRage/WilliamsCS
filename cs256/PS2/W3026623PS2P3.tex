\documentclass[10pt]{article}

% DO NOT EDIT THE LINES BETWEEN THE TWO LONG HORIZONTAL LINES

%---------------------------------------------------------------------------------------------------------

% Packages add extra functionality.
\usepackage{times,graphicx,epstopdf,fancyhdr,amsfonts,amsthm,amsmath,algorithm,xspace,hyperref}
\usepackage[left=1in,top=1in,right=1in,bottom=1in]{geometry}
\usepackage{sect sty}	%For centering section headings
\usepackage{enumerate}	%Allows more labeling options for enumerate environments 
\usepackage{epsfig}
\usepackage[space]{grffile}
\usepackage{algpseudocode}

% This will set LaTeX to look for figures in the same directory as the .tex file
\graphicspath{.} %The dot means current directory.

\pagestyle{fancy}

\lhead{\WilliamsID}
\chead{Problem Set \PSNumber \ --- Problem \ProblemNumber}
\rhead{\today}
\lfoot{CSci 256: Algorithm Design}
\cfoot{\thepage}
\rfoot{Spring 2018}

% Some commands for changing header and footer format
\renewcommand{\headrulewidth}{0.4pt}
\renewcommand{\headwidth}{\textwidth}
\renewcommand{\footrulewidth}{0.4pt}

% These let you use common environments
\newtheorem{claim}{Claim}
\newtheorem{definition}{Definition}
\newtheorem{theorem}{Theorem}
\newtheorem{lemma}{Lemma}
\newtheorem{observation}{Observation}
\newtheorem{question}{Question}

\setlength{\parindent}{0cm}


%---------------------------------------------------------------------------------------------------------

% DON'T CHANGE ANYTHING ABOVE HERE

% Edit below as instructed
\newcommand{\WilliamsID}{W3026623}	% Put you Williams ID in the braces
\newcommand{\PSNumber}{2}			% Put the problem set # in the braces
\newcommand{\ProblemNumber}{3}		% Put the problem # in the braces
\newcommand{\ProblemHeader}{Problem \ProblemNumber}	% Don't change this

\begin{document}

\vspace{\baselineskip}	% Add some vertical space
\textbf{I collaborated with:} W3023339


\vspace{\baselineskip}	% Add some vertical space
\textbf{Problem}

Until recently, there was no known method for computing the diameter of a graph that didn't first compute the shortest path between all pairs of nodes.  When graphs are dense, all-pairs shortest paths is fairly expensive, so some people have explored quicker algorithms which {\em estimate} the diameter of the graph.  Develop a linear-time algorithm that, given a graph $G$, returns a diameter estimate that is always within a factor of 1/2 of the true diameter.  That is, if the true diameter is $diam(G)$ then you should return a value $k$ where $diam(G)/2 \leq k \leq diam(G)$.\\
\textbf{Solution}
\begin{algorithm*}[h] 
	\caption{Diameter approximation of tree $T$ using vertex $r$} 
	\begin{algorithmic}
		\Function{Diameter}{$G$,$r$}
		\State Mark all $v \in V$ as unvisited
		\State Let $T$ be an empty graph
		\State Add $r$ to $T$; mark $r$ as visited; $r.level \leftarrow 0$
		\State $k\leftarrow 0$
		\While{There are $visited$ vertices} 
		\State{$current$ $\leftarrow$ some $visited$ vertex having minimum level} 
		\State $k \leftarrow current.level$ if $current.level > maxLevel$
		\State{Mark $current$ as explored}
		\For{$unvisited$ neighbors $v$ of $current$} 
		\State {Add $\{current,v\}$ to $T$} 
		\State {Mark $v$ as $visited$}
		\State{$v.level \leftarrow current.level + 1$}
		\EndFor
		\EndWhile\\
		\Return $k$
		\EndFunction
	\end{algorithmic}
\end{algorithm*} \\
Claim: The level $k$ shows the distance from a vertex $r$ to the vertex farthest away from it $t$ is $diam(G)/2 \leq k \leq diam(G)$.\\
All edges of $G$ not in $T$ connect vertices at consecutive levels
(or at the same level) of $T$, so $k$ represents the number of hops from $r$ to the node farthest away form it. For the lower bound of $k$, If there is more than one node in $T$ at level $k$ then the diameter of our graph is $2k$ or $diam(G)/2 = k$. Now for the upper bound, if we got lucky and picked an $r$ such that the shortest path from it to another vertex $t$ happend to be the longest path in $G$ then $k = diam(G)$.Thus, $diam(G)/2 \leq k \leq diam(G)$.
%End of feedback section

% DO NOT DELETE ANYTHING BELOW THIS LINE
\end{document}
