\documentclass[10pt]{article}

% DO NOT EDIT THE LINES BETWEEN THE TWO LONG HORIZONTAL LINES

%---------------------------------------------------------------------------------------------------------

% Packages add extra functionality.
\usepackage{times,graphicx,epstopdf,fancyhdr,amsfonts,amsthm,amsmath,algorithm,xspace,hyperref}
\usepackage[left=1in,top=1in,right=1in,bottom=1in]{geometry}
\usepackage{sect sty}	%For centering section headings
\usepackage{enumerate}	%Allows more labeling options for enumerate environments 
\usepackage{epsfig}
\usepackage[space]{grffile}
\usepackage{algpseudocode}

% This will set LaTeX to look for figures in the same directory as the .tex file
\graphicspath{.} %The dot means current directory.

\pagestyle{fancy}

\lhead{\WilliamsID}
\chead{Problem Set \PSNumber \ --- Problem \ProblemNumber}
\rhead{\today}
\lfoot{CSci 256: Algorithm Design}
\cfoot{\thepage}
\rfoot{Spring 2018}

% Some commands for changing header and footer format
\renewcommand{\headrulewidth}{0.4pt}
\renewcommand{\headwidth}{\textwidth}
\renewcommand{\footrulewidth}{0.4pt}

% These let you use common environments
\newtheorem{claim}{Claim}
\newtheorem{definition}{Definition}
\newtheorem{theorem}{Theorem}
\newtheorem{lemma}{Lemma}
\newtheorem{observation}{Observation}
\newtheorem{question}{Question}

\setlength{\parindent}{0cm}
\usepackage{tikz}
\usepackage{enumitem}
%---------------------------------------------------------------------------------------------------------

% DON'T CHANGE ANYTHING ABOVE HERE

% Edit below as instructed
\newcommand{\WilliamsID}{W3026623}	% Put you Williams ID in the braces
\newcommand{\PSNumber}{6}			% Put the problem set # in the braces
\newcommand{\ProblemNumber}{2}		% Put the problem # in the braces
\newcommand{\ProblemHeader}{Problem \ProblemNumber}	% Don't change this

\begin{document}

\vspace{\baselineskip}	% Add some vertical space
\textbf{I collaborated with: TA} 


\vspace{\baselineskip}	% Add some vertical space
\textbf{Solution}\\
The algorithm will turn this problem into a network flow and use the Ford-Fulkerson algorithm to determine whether every client can be connected simultaneously to a base station. We start by making a graph $G$ with a source $s$ and a sink $t$ vertex. We make every client and base station a vertex of $G$. Now, we go through every client and check if its distance from every base station is within $r$; if it is, we make an edge with capacity 1 from the client to the base station within distance. We then make an edge of capacity 1 from the source $s$ to every client. We also make edges from every base station to $t$ with capacities $L$. We have now turned this problem into a network flow. We now run the Ford-Fulkerson algorithm and if the max flow equals $n$ we return true, and return false if otherwise. \\
Time Complexity: Add every client and base station to $G: O(n+k)$ + Determine edges between clients and base stations $O(nk)$ + adding edges from $s$ to clients and from base stations to $t:O(n+k)$ + Ford-Fulkerson: $O(|E||V|C)= O(|E||V|C)$\\
Space Complexity: $G=(V,E)$ where $|V|=(n+k+2)$ and $|E|= (n+2n+k)$ in the worst case.
\begin{proof}
	Claim: If the maxflow equals $n$, then every client can be connected simultaneously to a base station.\\
	If the maxflow equals the number of clients $n$, then we know that edges from $s$ to every client has a flow of 1. We know that for every vertex, the flow coming into it has to equal the flow leaving it. Thus, having a max flow of $n$ means that every client is connected to a base station by the nature of our graph. This obeys the range and load conditions as clients are only connected to base stations within $r$ distance away and base stations have edges of capacity $L$ coming out of them. 
\end{proof}
%End of feedback section

% DO NOT DELETE ANYTHING BELOW THIS LINE
\end{document}
