\documentclass[10pt]{article}

% DO NOT EDIT THE LINES BETWEEN THE TWO LONG HORIZONTAL LINES

%---------------------------------------------------------------------------------------------------------

% Packages add extra functionality.
\usepackage{times,graphicx,epstopdf,fancyhdr,amsfonts,amsthm,amsmath,algorithm,xspace,hyperref}
\usepackage[left=1in,top=1in,right=1in,bottom=1in]{geometry}
\usepackage{sect sty}	%For centering section headings
\usepackage{enumerate}	%Allows more labeling options for enumerate environments 
\usepackage{epsfig}
\usepackage[space]{grffile}
\usepackage{algpseudocode}

% This will set LaTeX to look for figures in the same directory as the .tex file
\graphicspath{.} %The dot means current directory.

\pagestyle{fancy}

\lhead{\WilliamsID}
\chead{Problem Set \PSNumber \ --- Problem \ProblemNumber}
\rhead{\today}
\lfoot{CSci 256: Algorithm Design}
\cfoot{\thepage}
\rfoot{Spring 2018}

% Some commands for changing header and footer format
\renewcommand{\headrulewidth}{0.4pt}
\renewcommand{\headwidth}{\textwidth}
\renewcommand{\footrulewidth}{0.4pt}

% These let you use common environments
\newtheorem{claim}{Claim}
\newtheorem{definition}{Definition}
\newtheorem{theorem}{Theorem}
\newtheorem{lemma}{Lemma}
\newtheorem{observation}{Observation}
\newtheorem{question}{Question}

\setlength{\parindent}{0cm}
\usepackage{tikz}
\usepackage{enumitem}
%---------------------------------------------------------------------------------------------------------

% DON'T CHANGE ANYTHING ABOVE HERE

% Edit below as instructed
\newcommand{\WilliamsID}{W3026623}	% Put you Williams ID in the braces
\newcommand{\PSNumber}{6}			% Put the problem set # in the braces
\newcommand{\ProblemNumber}{3}		% Put the problem # in the braces
\newcommand{\ProblemHeader}{Problem \ProblemNumber}	% Don't change this

\begin{document}

\vspace{\baselineskip}	% Add some vertical space
\textbf{I collaborated with: TA} 


\vspace{\baselineskip}	% Add some vertical space
\textbf{Solution}\\
Let $G$ be a node-capacitated network. Let $G^{*}$ be an empty graph. For every node $v$ in $G$ add two vertices $v_{in}$ and $v_{out}$ to $G^{*}$ and an edge connecting them with capacity equal to $c_{v}$. If two vertices in $G$ are connected, then connect the corresponding pairs in $G^{*}$ with an edge of capacity equal to the capacity of the node where the original edge comes out of. Now, run the Ford–Fulkerson algorithm on $G^{*}$ to find an $s-t$ maximum flow in $G^{*}$ which corresponds to such flow in $G$. \\
Time Complexity: Building $G* = O(|V|+|E|)$ + Ford–Fulkerson algorithm $O(||E||V|CC)=O(|E||V|CC)$\\
Space Complexity: $O(|V|+|E|)$\\
\begin{proof}
	Claim: The analogue of the Max-Flow Min-Cut Theorem holds true.\\
	We can find a minimum $s-t$ cut in an edge-capacitated network using the Ford–Fulkerson algorithm. This cuts through edges of $G^{*}$ and separates the network. These edges that are cut correspond to vertices in $G$. By removing these vertices, we have found a minimum $s-t$ cut for the node-capacitated network $G$. We can now see there is a one-to-one correspondece between $G$ and $G^{*}$. The sum of the edges accross the minimum $s-t$ cut in $G^{*}$ corresponds to the maximum flow of $G^{*}$, so the sum of the capacities of the vertices in $G$ that correspond to these edges results in a maximum flow in $G$. 
\end{proof}
%End of feedback section

% DO NOT DELETE ANYTHING BELOW THIS LINE
\end{document}
