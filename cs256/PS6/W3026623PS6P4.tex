\documentclass[10pt]{article}

% DO NOT EDIT THE LINES BETWEEN THE TWO LONG HORIZONTAL LINES

%---------------------------------------------------------------------------------------------------------

% Packages add extra functionality.
\usepackage{times,graphicx,epstopdf,fancyhdr,amsfonts,amsthm,amsmath,algorithm,xspace,hyperref}
\usepackage[left=1in,top=1in,right=1in,bottom=1in]{geometry}
\usepackage{sect sty}	%For centering section headings
\usepackage{enumerate}	%Allows more labeling options for enumerate environments 
\usepackage{epsfig}
\usepackage[space]{grffile}
\usepackage{algpseudocode}

% This will set LaTeX to look for figures in the same directory as the .tex file
\graphicspath{.} %The dot means current directory.

\pagestyle{fancy}

\lhead{\WilliamsID}
\chead{Problem Set \PSNumber \ --- Problem \ProblemNumber}
\rhead{\today}
\lfoot{CSci 256: Algorithm Design}
\cfoot{\thepage}
\rfoot{Spring 2018}

% Some commands for changing header and footer format
\renewcommand{\headrulewidth}{0.4pt}
\renewcommand{\headwidth}{\textwidth}
\renewcommand{\footrulewidth}{0.4pt}

% These let you use common environments
\newtheorem{claim}{Claim}
\newtheorem{definition}{Definition}
\newtheorem{theorem}{Theorem}
\newtheorem{lemma}{Lemma}
\newtheorem{observation}{Observation}
\newtheorem{question}{Question}

\setlength{\parindent}{0cm}
\usepackage{tikz}
\usepackage{enumitem}
%---------------------------------------------------------------------------------------------------------

% DON'T CHANGE ANYTHING ABOVE HERE

% Edit below as instructed
\newcommand{\WilliamsID}{W3026623}	% Put you Williams ID in the braces
\newcommand{\PSNumber}{6}			% Put the problem set # in the braces
\newcommand{\ProblemNumber}{4}		% Put the problem # in the braces
\newcommand{\ProblemHeader}{Problem \ProblemNumber}	% Don't change this

\begin{document}

\vspace{\baselineskip}	% Add some vertical space
\textbf{I collaborated with: TA} 


\vspace{\baselineskip}	% Add some vertical space
\textbf{Solution}\\
Using Ford–Fulkerson on a network flow $G=(V,E)$, we can build a residual graph using the following rules, where $e=(u,v)\in E, e^{R}=(v,u),$ flow $f(e)$, capacity $c(e)$: \begin{gather*}
c_{f}(e) = \left\{\begin{array}{lr}
c(e)-f(e), & \text{if } e\in E\\
f(e), & \text{if } e^{R}\in E\\
\end{array}\right\} 
\end{gather*}
Now, grab all the vertices reachable from $s$ in the residual graph. These vertices are $upstream$. Then, switch the direction of every edge in $G$, and grab all the vertices reachable from $t$ in the residual graph. These vertices are $downstream$. All the remaining vertices are $central$.\\
Time Complexity: $O(|E||V|C)$\\
Space Complexity: $O(|E|+|V|)$\\
\begin{proof}
	Let $A$ be the set of all vertices reachable from $s$ in the residual graph of a network flow $G$. Let $U$ be the set of all $upstream$ vertices.\\
	Claim: $A = U$.\\
	In order to show $A=U$, we must show $U\subseteq A$ and $A\subseteq U$. We know $A$ is a min-cut because all edges not reachable from $s$ have maximum flow and are not part of a min-cut, thus all $upstream$ vertices are in $A$ (7.9 in the book). Therefore, $U\subseteq A$. Now, assume a vertex $v\in A$ is $downstream$. This means that $v$ is always on the sink side of every min-cut. However, we know $v$ is reachable from $s$. This is a contradiction, so $v$ is $upstream$ and $A\subseteq U$. We have showed that $U\subseteq A$ and $A\subseteq U$ which implies $A = U$.
\end{proof}
%End of feedback section

% DO NOT DELETE ANYTHING BELOW THIS LINE
\end{document}
