\documentclass[10pt]{article}

% DO NOT EDIT THE LINES BETWEEN THE TWO LONG HORIZONTAL LINES

%---------------------------------------------------------------------------------------------------------

% Packages add extra functionality.
\usepackage{times,graphicx,epstopdf,fancyhdr,amsfonts,amsthm,amsmath,algorithm,xspace,hyperref}
\usepackage[left=1in,top=1in,right=1in,bottom=1in]{geometry}
\usepackage{sect sty}	%For centering section headings
\usepackage{enumerate}	%Allows more labeling options for enumerate environments 
\usepackage{epsfig}
\usepackage[space]{grffile}
\usepackage{algpseudocode}

% This will set LaTeX to look for figures in the same directory as the .tex file
\graphicspath{.} %The dot means current directory.

\pagestyle{fancy}

\lhead{\WilliamsID}
\chead{Problem Set \PSNumber \ --- Problem \ProblemNumber}
\rhead{\today}
\lfoot{CSci 256: Algorithm Design}
\cfoot{\thepage}
\rfoot{Spring 2018}

% Some commands for changing header and footer format
\renewcommand{\headrulewidth}{0.4pt}
\renewcommand{\headwidth}{\textwidth}
\renewcommand{\footrulewidth}{0.4pt}

% These let you use common environments
\newtheorem{claim}{Claim}
\newtheorem{definition}{Definition}
\newtheorem{theorem}{Theorem}
\newtheorem{lemma}{Lemma}
\newtheorem{observation}{Observation}
\newtheorem{question}{Question}

\setlength{\parindent}{0cm}
\usepackage{tikz}
\usepackage{enumitem}
%---------------------------------------------------------------------------------------------------------

% DON'T CHANGE ANYTHING ABOVE HERE

% Edit below as instructed
\newcommand{\WilliamsID}{W3026623}	% Put you Williams ID in the braces
\newcommand{\PSNumber}{6}			% Put the problem set # in the braces
\newcommand{\ProblemNumber}{5}		% Put the problem # in the braces
\newcommand{\ProblemHeader}{Problem \ProblemNumber}	% Don't change this

\begin{document}

\vspace{\baselineskip}	% Add some vertical space
\textbf{I collaborated with: TA} 


\vspace{\baselineskip}	% Add some vertical space
\textbf{Solution}\\
This algorithm will use the algorithm from problem 4 (problem 23 in the book) to determine the set of $upstream, downstream,$ and $central$ vertices of the graph $G$. Now, if the set of $central$ vertices is empty, then $G$ has a unique minimum s-t cut.\\
Time Complexity: Problem 4 algorithm: $O(|E||V|C)$ + Check if there are $central$ vertices: $O(1)=O(|E||V|C)$\\
Space Complexity: The tree sets of vertices $upstream, downstream, \mbox{ and } central$, along with the graph.\\
Claim: Having no $central$ vertices ensure that there is a unique s-t cut of a network flow $G$. 
\begin{proof}
	$Central$ vertex $v$ is defined such that at least one minimum s-t cut $(A, B)$ for which $v \in A$, and at least one minimum s-t cut $(A′, B′)$ for which $v \in B′$. If there are no such vertices, then that means that all vertices are either $upstream$ or $downstream$. Furthermore, this means that all vertices are always part of the same s-t cut and always on their corresponding side; thus, there is only one s-t cut. 
\end{proof}
%End of feedback section

% DO NOT DELETE ANYTHING BELOW THIS LINE
\end{document}
