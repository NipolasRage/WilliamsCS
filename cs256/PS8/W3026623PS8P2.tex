\documentclass[10pt]{article}

% DO NOT EDIT THE LINES BETWEEN THE TWO LONG HORIZONTAL LINES

%---------------------------------------------------------------------------------------------------------

% Packages add extra functionality.
\usepackage{times,graphicx,epstopdf,fancyhdr,amsfonts,amsthm,amsmath,algorithm,xspace,hyperref}
\usepackage[left=1in,top=1in,right=1in,bottom=1in]{geometry}
\usepackage{sect sty}	%For centering section headings
\usepackage{enumerate}	%Allows more labeling options for enumerate environments 
\usepackage{epsfig}
\usepackage[space]{grffile}
\usepackage{algpseudocode}

% This will set LaTeX to look for figures in the same directory as the .tex file
\graphicspath{.} %The dot means current directory.

\pagestyle{fancy}

\lhead{\WilliamsID}
\chead{Problem Set \PSNumber \ --- Problem \ProblemNumber}
\rhead{\today}
\lfoot{CSci 256: Algorithm Design}
\cfoot{\thepage}
\rfoot{Spring 2018}

% Some commands for changing header and footer format
\renewcommand{\headrulewidth}{0.4pt}
\renewcommand{\headwidth}{\textwidth}
\renewcommand{\footrulewidth}{0.4pt}

% These let you use common environments
\newtheorem{claim}{Claim}
\newtheorem{definition}{Definition}
\newtheorem{theorem}{Theorem}
\newtheorem{lemma}{Lemma}
\newtheorem{observation}{Observation}
\newtheorem{question}{Question}

\setlength{\parindent}{0cm}
\usepackage{tikz}
\usepackage{enumitem}
%---------------------------------------------------------------------------------------------------------

% DON'T CHANGE ANYTHING ABOVE HERE

% Edit below as instructed
\newcommand{\WilliamsID}{W3026623}	% Put you Williams ID in the braces
\newcommand{\PSNumber}{8}			% Put the problem set # in the braces
\newcommand{\ProblemNumber}{2}		% Put the problem # in the braces
\newcommand{\ProblemHeader}{Problem \ProblemNumber}	% Don't change this

\begin{document}

\vspace{\baselineskip}	% Add some vertical space
\textbf{I collaborated with:} went to TA sessions 


\vspace{\baselineskip}	% Add some vertical space
\textbf{Question}\\
Chapter 10, Problem 1. Hint: Try the idea we used for vertex cover (see Property 10.3):  Delete some element of some $B_i$ (and $B_i$) from the problem instance  Think about how this reduction to a smaller problem can help.\\
\textbf{Solution}\\
Let $I=\{B_{1},...,B_{m}\}$ be the collection of subsets of the set $A$ and $k$ be the maximum size of the Hitting Set. Our base cases are if $I$ is empty then we return true, and if $k==0$ then we return false. Let $b_{i}\in B_{i}$. We remove every element of $I$ that contains $b_{i}$ and decrease $k$ by one. This forms the recursive call. Where the parameters are the collection of subsets $I$ and $k$. We do this for all $c$ elements of every $B_{i}$. If at any point true is returned then there exists a hitting set of size at most $k$. However, if false is returned then no such set exists.\\
Time Complexity: There are at most $c$ recursive calls for each value of $k$, so $c^{k}$ nodes. Removing sets require visiting at most $m$ sets and checking at most $c$ elements in every set. So $O(c^{k}* cm)=O(m)$.\\
\begin{proof}
	Base Case: If there are no subsets of $A$, then our Hitting Set has hit all elements of $I$. Also, if $k=0$ then this means that there are still subsets in $I$ not being hit by our "Hitting Set."\\
	Inductive Hypothesis: Assume there is a hitting set of size $k-1$ for a collection of subsets of $A$.\\
	Inductive Step: Let $H$ be such hitting set of size $k-1$ for a collection of subsets of $A$. If we add an element to $H$ then it also hits subsets of $A$ containing this element. Thus, $H$ is still a hitting set. Now, suppose $H$ was not a hitting set for a collection of subsets. Then, even if we add an element to $H$ and add subsets containing said elements, $H$ will not be a hitting set because the original subsets are still not hit by the elements of $H$.
\end{proof}
%End of feedback section

% DO NOT DELETE ANYTHING BELOW THIS LINE
\end{document}
