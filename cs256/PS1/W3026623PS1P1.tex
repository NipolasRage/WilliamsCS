\documentclass[10pt]{article}

% DO NOT EDIT THE LINES BETWEEN THE TWO LONG HORIZONTAL LINES

%---------------------------------------------------------------------------------------------------------

% Packages add extra functionality.
\usepackage{times,graphicx,epstopdf,fancyhdr,amsfonts,amsthm,amsmath,algorithm,algorithmic,xspace,hyperref}
\usepackage[left=1in,top=1in,right=1in,bottom=1in]{geometry}
\usepackage{sect sty}	%For centering section headings
\usepackage{enumerate}	%Allows more labeling options for enumerate environments 
\usepackage{epsfig}
\usepackage[space]{grffile}

% This will set LaTeX to look for figures in the same directory as the .tex file
\graphicspath{.} %The dot means current directory.

\pagestyle{fancy}

\lhead{\WilliamsID}
\chead{Problem Set \PSNumber \ --- Problem \ProblemNumber}
\rhead{\today}
\lfoot{CSci 256: Algorithm Design}
\cfoot{\thepage}
\rfoot{Spring 2018}

% Some commands for changing header and footer format
\renewcommand{\headrulewidth}{0.4pt}
\renewcommand{\headwidth}{\textwidth}
\renewcommand{\footrulewidth}{0.4pt}

% These let you use common environments
\newtheorem{claim}{Claim}
\newtheorem{definition}{Definition}
\newtheorem{theorem}{Theorem}
\newtheorem{lemma}{Lemma}
\newtheorem{observation}{Observation}
\newtheorem{question}{Question}

\setlength{\parindent}{0cm}


%---------------------------------------------------------------------------------------------------------

% DON'T CHANGE ANYTHING ABOVE HERE

% Edit below as instructed
\newcommand{\WilliamsID}{W3026623}	% Put you Williams ID in the braces
\newcommand{\PSNumber}{1}			% Put the problem set # in the braces
\newcommand{\ProblemNumber}{1}		% Put the problem # in the braces
\newcommand{\ProblemHeader}{Problem \ProblemNumber}	% Don't change this

\begin{document}

\vspace{\baselineskip}	% Add some vertical space
\textbf{I collaborated with:} Nevin Bernet

\vspace{\baselineskip}	% Add some vertical space
\textbf{Notes to Instructor or TA:} If needed, include here any special notes for TAs or instructor; delete if no notes

\vspace{\baselineskip}	% Add some vertical space
\textbf{Problem}

Take the following list of functions and arrange them in ascending order of growth rate.  That is, if function $g(n)$ immediately follows function $f(n)$ in your list, then it should be the case that $f(n)$ is $O(g(n))$.  Please prove your claims.
\begin{enumerate}
	\item $f_{1}(n) = 2^{\sqrt{(\log n)}}$
	\item $f_{2}(n)=2^{n}$
	\item $f_{3}(n)=n^{4/3} = 2^{4/3 \log n}$
	\item $f_{4}(n)=n(\log n)^{3}$
	\item $f_{5}(n)=n^{\log n} = 2^{(\log n )^{2}}$
	\item $f_{6}(n)=2^{2^{n}}$
	\item $f_{7}(n)=2^{n^{2}}$
\end{enumerate}
  
\textbf{Solution}
\begin{enumerate}
	\item $f_{1}(n) = 2^{\sqrt{(\log n)}}$
	\item $f_{4}(n)=n(\log n)^{3}:$ $\sqrt{(\log n)}$ grows very slowly, so the 2 in $f_{1}$ will be raised to small exponents. 
	\item $f_{3}(n)=n^{4/3} = 2^{4/3 \log n}:$ $\lim_{n\to\infty} \frac{n(\log n)^{3}}{n^{4/3}} = \lim_{n\to\infty} \frac{(\log n)^{3}}{n^{1/3}} = \lim_{n\to\infty} \frac{(\log n)}{n^{1/9}} = 0$ as exponentials grow faster than logarithms (shown in class).
	\item $f_{5}(n)=n^{\log n} = 2^{(\log n )^{2}}:$ Assuming this is in base 2, $(\log n)^{2}$ grows faster than $4/3 \log n$ when $n>2^{4/3}$. 
	\item $f_{2}(n)=2^{n}:$ $n$ grows faster than $(\log n)^{2}$.
	\item $f_{7}(n)=2^{n^{2}}:$ $n^{2}$ grows faster than $n$.
	\item $f_{6}(n)=2^{2^{n}}:$ $2^{n}$ grows faster than $n^{2}$ (discussed in class).
\end{enumerate}


%End of feedback section

% DO NOT DELETE ANYTHING BELOW THIS LINE
\end{document}
