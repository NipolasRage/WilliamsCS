\documentclass[10pt]{article}

% DO NOT EDIT THE LINES BETWEEN THE TWO LONG HORIZONTAL LINES

%---------------------------------------------------------------------------------------------------------

% Packages add extra functionality.
\usepackage{times,graphicx,epstopdf,fancyhdr,amsfonts,amsthm,amsmath,algorithm,algorithmic,xspace,hyperref}
\usepackage[left=1in,top=1in,right=1in,bottom=1in]{geometry}
\usepackage{sect sty}	%For centering section headings
\usepackage{enumerate}	%Allows more labeling options for enumerate environments 
\usepackage{enumitem}
\usepackage{epsfig}
\usepackage[space]{grffile}

% This will set LaTeX to look for figures in the same directory as the .tex file
\graphicspath{.} %The dot means current directory.

\pagestyle{fancy}

\lhead{\WilliamsID}
\chead{Problem Set \PSNumber \ --- Problem \ProblemNumber}
\rhead{\today}
\lfoot{CSci 256: Algorithm Design}
\cfoot{\thepage}
\rfoot{Spring 2018}

% Some commands for changing header and footer format
\renewcommand{\headrulewidth}{0.4pt}
\renewcommand{\headwidth}{\textwidth}
\renewcommand{\footrulewidth}{0.4pt}

% These let you use common environments
\newtheorem{claim}{Claim}
\newtheorem{definition}{Definition}
\newtheorem{theorem}{Theorem}
\newtheorem{lemma}{Lemma}
\newtheorem{observation}{Observation}
\newtheorem{question}{Question}

\setlength{\parindent}{0cm}


%---------------------------------------------------------------------------------------------------------

% DON'T CHANGE ANYTHING ABOVE HERE

% Edit below as instructed
\newcommand{\WilliamsID}{W3026623}	% Put you Williams ID in the braces
\newcommand{\PSNumber}{1}			% Put the problem set # in the braces
\newcommand{\ProblemNumber}{4}		% Put the problem # in the braces
\newcommand{\ProblemHeader}{Problem \ProblemNumber}	% Don't change this

\begin{document}

\vspace{\baselineskip}	% Add some vertical space
\textbf{I collaborated with:} Nevin Bernet

\vspace{\baselineskip}	% Add some vertical space

\vspace{\baselineskip}	% Add some vertical space
\textbf{Problem}

The National Resident Matching Program (NRMP) matches medical students with residency programs.  Here is what the NRMP says on their website:
\begin{quote}
	The NRMP conducts its Matches using a mathematical algorithm that pairs the rank ordered preferences of applicants and program directors to produce a Òbest fitÓ for filling available training positions. Research on the NRMP algorithm was a basis for Dr,  Alvin RothÕs receipt of the 2012 Nobel Prize in Economics.
\end{quote}
In this matching problem, there are $n$ students and $m$ hospitals.  Each hospital $h_{i}$ has $p_{i}$ available positions.  Each student ranks the $m$ hospitals and each hospital ranks the $n$ students.  Since there are more students than total positions available, we assume that $n > \sum_{i=1}^{m} p_{i}$.  Thus, some students are never matched.   As a result, we need a slightly expanded version of stability.  As before, the matching is unstable if 
\begin{itemize}
	\item $s$ is assigned to $h$ and $s'$ is is assigned to $h'$ but $s$ prefers $h'$ to $h$ and $h'$ prefers $s$ to $s'$.  
\end{itemize}
But it is also unstable if 
\begin{itemize}
	\item $s$ is assigned to $h$ and $s'$ is not assigned to a hospital but $h$ prefers $s'$ to $s$.
\end{itemize}
Give an algorithm to find a stable matching of students to hospitals where every hospital position is filled with a student.  Show that your algorithm is correct and that it runs in time polynomial in $n$ and $m$.  Your algorithm description and analysis should be clear and concise.  
  
\textbf{Solution}
\begin{algorithm*}[h] 
	\caption{Stable matching} 
	\begin{algorithmic}[1] 
		\REQUIRE set of lists of every student's preference and another set of lists of every hospital's preference. 
		\medskip 
		\STATE initialize each student and hospital to be free
		\WHILE{some student is free and hasn't "proposed" to every hospital}
		\STATE $s \leftarrow$ such student
		\STATE $h_{i}\leftarrow$ 1st hospital on $s$'s list to whom $s$ has not yet proposed 
		\IF {$h_{i}$ still has available positions}
		\STATE match $s$ to $h_{i}$
		\ELSIF {$h_{i}$ prefers $s$ to another student already matched in the hospital}
		\STATE swap least preferable student with $s$
		\ELSE
		\STATE $h_{i}$ rejects $s$
		\ENDIF
		\ENDWHILE
	\end{algorithmic} 
\end{algorithm*} 
\\Space: n lists of length n,  m lists of length m, m heaps of size $p_{i}$: $O(n^{2}+m^{2}+mp_{i})$
\\Time: The selection of students, a hospital's ranking of a given student and vice versa, are almost identical to the P-R algorithm which we discussed were all constant time operations. Heaps allow us to check the least preferred student currently in a hospital in constant type. Swapping students takes $O(\log (n))$ time as the new student added needs to percolate down to their spot in the heap based on the hospital's preference.
\\Overall Running Time: 
\begin{description}
	\item $\bullet$ Preprocessing: $O(n^{2})$ for lists + $O(n)$ for queue of students.
	\item $\bullet$ Loop: $O(n^2)*O(1)*O(\log (n))$ 
\end{description}
Suppose, for the sake of contradiction, that the algorithm proposed above does not produce a stable matching. Then, the final matching has an instability \{$s,h$\}. That is, $s$ is matched to $h$' but $s$ prefers $h$ over $h$'. Also, $h$ is matched to $s$' but $h$ prefers $s$ over $s$'. According to the algorithm, $s$ proposed to $h$ before proposing to $h$'. Therefore, $h$ either removed $s$ from its heap, or $s$ did not rank high enough to get in the heap. However, hospitals always choose the most preferred candidate, so $h$ prefers $s$' over $s$. This contradicts the instability defined above, so our algorithm produces a stable matching.
%End of feedback section

% DO NOT DELETE ANYTHING BELOW THIS LINE
\end{document}
