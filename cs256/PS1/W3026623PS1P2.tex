\documentclass[10pt]{article}

% DO NOT EDIT THE LINES BETWEEN THE TWO LONG HORIZONTAL LINES

%---------------------------------------------------------------------------------------------------------

% Packages add extra functionality.
\usepackage{times,graphicx,epstopdf,fancyhdr,amsfonts,amsthm,amsmath,algorithm,algorithmic,xspace,hyperref}
\usepackage[left=1in,top=1in,right=1in,bottom=1in]{geometry}
\usepackage{sect sty}	%For centering section headings
\usepackage{enumerate}	%Allows more labeling options for enumerate environments 
\usepackage{epsfig}
\usepackage[space]{grffile}

% This will set LaTeX to look for figures in the same directory as the .tex file
\graphicspath{.} %The dot means current directory.

\pagestyle{fancy}

\lhead{\WilliamsID}
\chead{Problem Set \PSNumber \ --- Problem \ProblemNumber}
\rhead{\today}
\lfoot{CSci 256: Algorithm Design}
\cfoot{\thepage}
\rfoot{Spring 2018}

% Some commands for changing header and footer format
\renewcommand{\headrulewidth}{0.4pt}
\renewcommand{\headwidth}{\textwidth}
\renewcommand{\footrulewidth}{0.4pt}

% These let you use common environments
\newtheorem{claim}{Claim}
\newtheorem{definition}{Definition}
\newtheorem{theorem}{Theorem}
\newtheorem{lemma}{Lemma}
\newtheorem{observation}{Observation}
\newtheorem{question}{Question}

\setlength{\parindent}{0cm}


%---------------------------------------------------------------------------------------------------------

% DON'T CHANGE ANYTHING ABOVE HERE

% Edit below as instructed
\newcommand{\WilliamsID}{W3026623}	% Put you Williams ID in the braces
\newcommand{\PSNumber}{1}			% Put the problem set # in the braces
\newcommand{\ProblemNumber}{2}		% Put the problem # in the braces
\newcommand{\ProblemHeader}{Problem \ProblemNumber}	% Don't change this

\begin{document}

\vspace{\baselineskip}	% Add some vertical space
\textbf{I collaborated with:} Nevin Bernet

\vspace{\baselineskip}	% Add some vertical space
\textbf{Notes to Instructor or TA:} If needed, include here any special notes for TAs or instructor; delete if no notes

\vspace{\baselineskip}	% Add some vertical space
\textbf{Problem}

Decide whether you think the following statements are true or false.  If a statement is true, give a short explanation why it's the case.  If it's false, give a counterexample.
\begin{description}
	\item [(a)] In every instance of the Stable Matching Problem there is a stable matching containing a pair $(m,w)$ such that $m$ was ranked first on the preference list of $w$ and $w$ was ranked first on the preference list of $m$.\\
	
	\item [(b)] Consider an instance of the Stable Matching Problem in which there exists a man $m$ and a woman $w$ such that $m$ is ranked first on the preference list of $w$ and $w$ is ranked first on the preference list of $m$.  Then the pair $(m,w)$ belongs to every possible stable matching
	for this instance. \\
	
\end{description}
  
\textbf{Solution}
\begin{enumerate}
	\item False. Let the following table be an input to the Stable Matching Problem\\
	\begin{table}[h]
		\begin{center}
			\begin{tabular}{r|ccc}
				Name & \multicolumn{3}{c}{Preference}\\
				\hline
				Alice & Bob & Don \\
				Bertha & Don & Bob  \\
				Bob & Bertha & Alice  \\
				Don & Alice & Bertha 
			\end{tabular}
		\end{center}
	\end{table}\\
In this case, a stable matching will be made even though no two people have each other as their top preference.
	\item True, a stable matching is defined as one where for every unmatched $\{m, w\}$, either $m$ prefers his match to $w$, or $w$ prefers her match to $m$. Using the P-R algorithm, the man will propose to his first choice, and the woman will be engaged and never trade down, or trade up to get with him if she is taken already.
\end{enumerate}


%End of feedback section

% DO NOT DELETE ANYTHING BELOW THIS LINE
\end{document}
