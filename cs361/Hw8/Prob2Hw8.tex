% !TEX root = Hw8Template.tex

An unrestricted grammar is a quadruple $G=(V,\Sigma,R,S)$ where
\begin{itemize}
	\item $V$ is a finite set of non-terminal symbols;
	\item $\Sigma$ is a finite set of terminal symbols disjoint from $V$;
	\item $S \in V$ is the unique start symbol.
	\item $R \subset (V \cup \Sigma)^*(V)(V \cup \Sigma)^* \times (V \cup \Sigma)^*$ is a set of rules .
\end{itemize}
The only difference between an unrestricted grammar and a context-free grammar is in the rules.  Rules in context-free grammars have a single non-terminal on the left hand side, whereas rules in an unrestricted grammar may have any string of terminals and non-terminals on the left side, 
but must include at least one non-terminal. As with context-free grammars, although rules are formally
tuples, they are conventionally written with an arrow separating the right and left sides.

Derivations are similar to context-free grammars except the definition of ``yields'' is slightly different: 
we may substitute the right hand side of any rule into a derivation if the derivation has a substring matching the left hand side of the rule.  
More formally: \begin{description}
      \item[Yields]
      Given an unrestricted grammar, G = $(V , \Sigma , S, R)$, and two strings $x$ and $y$
      in $(V \cup \Sigma)^*$ such that $x = \alpha{}\sigma{}\beta$ and 
      $y = \alpha \gamma \beta$ where 
      $\alpha, \sigma, \gamma, \beta \in (V \cup \Sigma)^*$ and $( \sigma, \gamma) \in R$
      we say that $x$ {\em yields (or directly derives)} $y$.  In this case we write
       \[
         x \Longrightarrow y
       \]
 	\end{description}


For example, here's a grammar that generates the language $\{a^{n}b^{n}c^{n} \,|\, n \geq 1\}$:

\begin{center}
\begin{tabular}{llcccc}
S & $\rightarrow$ & ABCS & $|$ & $T_{c}$ \\
CA & $\rightarrow$ & AC \\
BA & $\rightarrow$ & AB \\
CB & $\rightarrow$ & BC \\
C$T_{c}$ & $\rightarrow$ & $T_{c}$c & $|$ & $T_{b}$c \\
B$T_{b}$ & $\rightarrow$ & $T_{b}$b & $|$ & $T_{a}$b \\
A$T_{a}$ & $\rightarrow$ & $T_{a}$a \\
$T_{a}$ & $\rightarrow$ & $\varepsilon$
\end{tabular}
\end{center}
%%%%%%% end longform

Let $L \subseteq \Sigma^{*}$ be a language generated by an unrestricted grammar $G$.  Show that $L$ is Turing recognizable using the result of problem 1 (even if you were unable to complete problem 1).  

\longform{%%%%%%
Hint: The restriction that you must use problem 1 is actually intended as a hint that may help you
solve problem 1. Finding a concrete example of a language ``$D$'' that fits the requirements on
$D$ in problem 1 in the case that $C$ is a language described by a context-sensitive grammar should
(I hope) give you a concrete sense of how a string in the language $D$ might serve as a witness
of membership in $C$.
}%%%%%%%%%

	

\begin{solution}

% Place your answers here
Given $C = L = L(G)$ consider the language:
\[
D = \{w\#w_0\#w_1\#...\#w_n ~|~ w\in L, w = S, ~ \forall ~ i<n ~ w_i \Longrightarrow w_{i+1}\}
\]

$D$ represents the language of strings $w$ in $L$ followed by derivations that lead to the corresponding strings $w$ using the rules of $G$. Each derivation is separated by a $\#$. $D$ is of the form 
\[
C = \{ x \,|\, \mbox{there exists $y$ such that $x \# y  \in D$ } \}.
\]

as every $x\in C$ is the first substring of $D$. In order to show $C$ is Turing-recognizable, we must show $D$ is Turing -decidable (this is using our result from problem 1). We can build a turing machine $T_D$ that decides $D$ by first validating our string. That is, $w$ and $w_n$ must equal to each other and contain only terminals, every $w_i$ must contain symbols in either $\Sigma \mbox{ or } V$, and $w_0$ must be equal to the start state of the grammar. $T_D$ will then go through the start state of $G$ and verify that it can get to $w_1$, so on and so forth. If there is ever the case that $w_i \not \Longrightarrow w_{i+1}$, then $T_D$ rejects the input. Once $w_n$ is reached, then $T_D$ accepts the input. This machine will always halt as a string of terminals produced by $G$ will always be derived in a finite number of steps, thus $w$ and the derivations $w_0$ to $w_n$ are finite. The cases where $T_C$ would not halt are when the input is not produced by the grammar and $T_D$ will try an infinite number of configurations as input.
\end{solution}
