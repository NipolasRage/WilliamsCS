% !TEX root = Hw8Template.tex

There are five remaining problem statements for this assignment. Each of the remaining problems describes
a question informally. Three of these questions are decidable and two are not. For each of the questions
that is decidable, you should formulate the question as a language and then argue that there is a Turing
machine that decides this language. You don't have to turn in anything for the other two questions.
To ensure that it is clear which questions you are answering, make sure that the number of each problem
you decide to answer is clearly visible on the page(s) containing your answer. 

For example, if one of the questions below was ``Does the language of a given 
finite automaton contain a specific string $w$?'' and you believed that this language
was decidable (it is!), you would formulate this as the language
\[
A_{DFA} = \{ \langle D, w \rangle \,|\, \mbox{$D$ is a DFA that accepts $w$}\}
\]
and then explain how a Turing machine could use the description of $D$ provided on its input
to simulate $D$ on $w$ and determine whether it reached a final state.

\item
\begin{quote}
Given a Turing machine $M$ whose encoding requires $l$ symbols, does $M$ run for at least $l$
steps on all possible inputs?
\end{quote}

\begin{solution}

% Place your answers here
$A_{STEP} = \{\langle M \rangle ~|~ M \mbox{ is a Turing Machine and } M \mbox{ runs for at least } |\langle M \rangle|$ steps for a given input$\}$.\\
We can build a turing machine that simulates $M$ on all inputs of length up to $|\langle M \rangle|$. If $M$ accepts or rejects any strings before $|\langle M \rangle|$ steps then we reject the machine. However, if $M$ does not halt at $|\langle M \rangle|-1$ steps for any input up to length $|\langle M \rangle|$, then we accept the machine. This is because all subsequent strings have one of the tested strings as a prefix, so they must take at least $|\langle M \rangle|$ steps.

\end{solution}
