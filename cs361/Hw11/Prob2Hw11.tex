% !TEX root = Hw11Template.tex

\begin{solution}

% Place your answers here
\begin{itemize}
	\item [a]) Assume that $P=NP$ for the sake of contradiction. We know that $L\in P$ given that $L\in NP$. This means that an instance of $L$ can be verified in polynomial time and that there is some polynomial-time algorithm for solving it. Since $L$ is not NP-complete then we know an NP-complete problem cannot be reduced to $L$. This means that an instance of $L$ cannot be solved in polynomial time to solve an instance of any NP-complete problem. However, this violates our statement that there is some polynomial-time algorithm for solving $L$, so our assumption that $P=NP$ is incorrect. Therefore, if there exists an $L\in NP$ that is not NP-complete then it must be the case that $P\ne NP$.
\end{itemize}

\end{solution}
